\clearpage
\chapter{Introduction}
\label{ch:introduction}
In his 1963 book, \textit{Formalized Music}, Composer, Engineer, and
Architect, Iannis Xenakis described the foundation for his own
reinterpretation of conventional music theory:
\begin{quotation}
  ``All sound is an integration of grains, of elementary sonic
  particles, of sonic quanta. Each of these elementary grains has a
  threefold nature: duration, frequency, and intensity.''
\end{quotation}
Instead of using high-level musical concepts like pitch and meter to
compose music, Xenakis posited, only the three elementary qualities
(frequency, duration and intensity) are necessary. Because it is
impractical to describe sounds as the sum of hundreds or thousands of
elementary sonic particles, he proposed that statistical and probabilistic
models should be used to define sounds at the macroscopic
level. Additionally, similar mathematical models should describe other
compositional concepts including rhythm, form, and melody. Xenakis
named this new style \textit{stochastic}, and he considered it a
generalization of existing music theory: High-level musical constructs
such as melody, harmony, and meter are mathematical abstractions of
the elementary sonic particles, and alternative abstractions such as
non standard tuning should exist as equals within the same mathematical
framework. Music composition, he claimed, requires a \emph{deep}
understanding of the mathematical relationship between sonic elements
and musical abstractions, and music composition necessarily involves
the formulation of new high-level musical constructs built from the
low-level elements.

\paragraph{Music and Audio Engineering}
% \label{sec:audio-engineering}
Today, just over 50 years after \textit{Formalized Music} was first
published, some of Xenakis' ideals are widely accepted, while others
are largely ignored. As computers, amplifiers, and electronics become
ubiquitous in the composition, production, and performance of music,
mathematics and music necessarily become more interconnected, and the
line between musician and engineer becomes increasingly indistinct.
As a result, a \textit{deep} understanding of the mathematics of music
is also increasingly valuable to musicians.  The most common tools for
shaping sounds typically use mathematical language such as frequency,
milliseconds, and decibels, requiring us to translate between
mathematical and musical concepts. The controls for electronic
synthesizers use words that are synonyms for Xenakis' three sonic
elements (duration, frequency, and intensity). While countless
high-level interfaces for manipulating sounds have been designed, it
would appear that the most useful (and the most widely used) are the
simplest.
\begin{enumerate}
\item The equalizer, a frequency specific amplifier
\item The delay, a temporal shift on the order of milliseconds
\item The compressor, an automatic gain control
\end{enumerate}

\paragraph{Sound and Space}
%\label{sec:sound-space}
It is curious that the most useful tools for engineering sound would
also be the simplest: The compositional equivalent would be composing
by starting with pure sine waves as Xenakis originally suggested.
From an engineering perspective, Xenakis' three sonic elements are
rational choices. By summing together the correct recipe of sine
tones, we can construct any audio waveform. However, a waveform is not
the same as music, or even the same as sound. Sound is
three-dimensional, sound has a direction, and sound exists in
space. Could space be the missing sonic element in electronic audio
production? If we design our audio engineering tools such that space
is considered an equal to frequency and intensity, can we build
high-level tools that are as effective as the low level tools we depend
on, or even as effective as acoustic instruments? The three projects
described in this thesis are directly inspired by Iannis Xenakis, and rest on
the foundation of stochastic music: \refmod, \polytempic, and
\thesis. Each project builds on existing paradigms in composition or
audio engineering, and together they treat space and time as equals,
and as true elements of music.

\section{\refmod}
\label{sec:refmod-intro}
\newthought {Music and space} are intimately connected. The first
project, described in \autoref{ch:ref-mod}, explores how we can
compose music using acoustic reflections in architectural space as a
medium. \refmod is a software tool that lets us design and experiment
with abstract acoustic lenses or ``sound mirrors'' in two
dimensions. It is directly inspired by the music and architecture of
Xenakis.

\section{\polytempic}
\label{sec:polytempic-intro}

\newthought{Music and Time} are inseparable. All music flows through
time and depends on temporal constructs - the most common being meter
and tempo. Accelerating or decelerating tempi are common in many
styles of music, as are polyrhythms.  Music with multiple simultaneous
tempi or \textit{polytempic music} is less common, but still many
examples can be found. Fewer examples of music with simultaneous tempi
that shift relative to each other exist, however, and it is difficult
for musicians to accurately perform changing tempi in
parallel. Software is an obvious choice for composing complex and
challenging rhythms such as these, but existing compositional software
makes this difficult. \polytempic offers a solution to this challenge
by describing a strategy for composing music with multiple
simultaneous tempi that accelerate and decelerate relative to each
other. In \autoref{ch:polytempic} we derive an equation for smoothly
ramping tempi to converge and diverge as musical events within a
score, and show how this equation can be used as a stochastic process
to compose previously inaccessible sonorities.

\section{\thesis}
\label{sec:hypercompression-intro}
\newthought{Time and space} are the means and medium of music. \thesis
explores a new tool built for shaping music in time and space. The
tool build on the dynamic range compression paradigm.  We usually
think of compression in terms of \emph{reduction}: We use data
compression to reduce bit-rates and file sizes and audio compression
to reduce dynamic range. Record labels' use of dynamic range
compression as a weapon in the \emph{loudness
  war}\sidenote[][]{Beginning in the 1990s, record labels have
  attempted to make their music louder than the music released by
  competing labels. ``Loudness War'' is the popular name given to the
  trend of labels trying to out-do each other at the expense of audio
  fidelity.}\cite{Deruty2014a} has resulted in some of today's music
recordings utilizing no more dynamic range than a 1909 Edison
cylinder.\cite{Katz2007} A deeper study of dynamic range compression,
however, reveals more subtle and artistic applications beyond that of
reduction. A skilled audio engineer can apply compression to improve
intelligibility, augment articulation, smooth a performance, shape
transients, extract ambience, de-ess vocals, balance multiple signals,
or even add distortion.\cite{Case2007} At its best, the compressor is
a tool for temporal shaping, rather than a tool for dynamic reduction.

\thesis expands the traditional model of a dynamic range compressor to
include spatial shaping.  Converting measurement of sound from the
cycles per second (in the temporal domain) to wavelength (in the
spatial domain) is a common objective in acoustics and audio
engineering practices.\cite{Davis1989} While unconventional, spatial
processing is a natural fit for the compression model. The mathematics
and implementation of the Hypercompressor are described in detail in
\autoref{ch:hypercompressor}.

%\paragraph{Performance}
\thesis was used in the live performance of \textit{De
  L'Exp\'{e}rience}, a new musical work by composer Tod Machover for
narrator, organ, and electronics. During the premier at the Maison
Symphonique de Montr\'{e}al in Canada, \thesis was used to blend the
electronics with the organ and the acoustic space. 
A detailed description of how \thesis featured in this performance is
also discussed in \autoref{ch:hypercompressor}.


\section{Universality}
\label{sec:universality}
At the MIT Media Lab, we celebrate the study and practice of projects
that exist outside of established academic disciplines. The Media Lab
(and the media) have described this approach as interdisciplinary,
cross-disciplinary, anti-disciplinary, or post-disciplinary; rejecting
the clich\'{e} that academics must narrowly focus their studies
learning \textit{more and more about less and less}, and eventually
knowing \textit{everything about nothing}.  The projects described
here uphold the vision of both Xenakis and the Media Lab. Each chapter
documents the motivations and implementation of a new tool for
manipulating space and sound. Each project draws from an assortment of
fields including music, mathematics, computer science, acoustics,
audio engineering and mixing, sound reinforcement, multimedia
production, and live performance. 

% How can we describe and document a project with such broad subject
% material? Within a single discipline, there is an accepted hierarchy
% of concepts, and we are expected to develop a \emph{deep}
% understanding that penetrates this hierarchy. We expect students to be
% literate in algebra, geometry and calculus before studying
% physics. When we describe a physics problem, we depend on an
% established collection of language, notation, and theory.

% This example reveals the curious tension between breadth and depth:
% The \textit{depth} of a disciplinary approach provides the language
% and abstraction that enable us to describe content and communicate at
% a high level. Depth is essential for solving non-trivial
% problems. However, solutions to the most complex and interesting
% real-world projects always span multiple disciplines. It appears we
% need breadth \emph{and} depth simultaneously. The impact of Iannis
% Xenakis and the success of the Philips Pavilion illustrate the
% efficacy of this approach. 

% \TODO{this thesis is an experiment in breadth and depth}

% One of the goals of this thesis is to
% bridge disciplines by describing the material in a way that is
% accessible to readers that are not experts in all the fields
% involved.


%%% Local Variables:
%%% mode: latex
%%% TeX-master: "CharlesHolbrow_MAS_Thesis"
%%% End:
