\clearpage
\openAFM{\thesis}{\subtitle}{Charles J. Holbrow}

\begin{raggedright}
\noindent Submitted to the Program~in~Media~Arts~and~Sciences,\\
School~of~Architecture~and~Planning, in partial fulfillment\\
of the requirements for the degree of\\
\textbf{Master~of~Science in~Media~Arts~and~Sciences}\\
at the \textbf{Massachusetts~Institute~of~Technology} \\
\noindent September 2015\\
\noindent \textcircled{c}~2015~Massachusetts~Institute~of~Technology. All rights reserved.\\
\end{raggedright}

\begingroup
\let\clearpage\relax
\let\cleardoublepage\relax
\let\cleardoublepage\relax
\chapter*{Abstract}
\label{ch:abstract}
The theory of stochastic music proposes that we think of music as a
vertical integration of mathematics, the physics of sound,
psychoacoustics, and traditional music theory. In \textit{\thesis,
  \subtitle} we explore the design and implementation of three
innovative musical projects that build on a deep vertical integration
of science and technology in different ways: \polytempic, \refmod, and
\thesis. \polytempic proposes a mathematical approach for composing
previously inaccessible polytempic music. The \refmod introduces an
interface for quickly sketching abstract architectural and musical
ideas. \thesis describes new technique for manipulating music in space
and time. For each project, we examine how stochapstic theory can help
us discover and explore new musical possibilites, and we discuss the
advantages and shortcomings of this approach.
\endgroup

\vspace{10mm}
\begin{tabbing}
Thesis Supervisor: \=   TOD MACHOVER \\
\> Muriel R. Cooper Professor of Music and Media \\ 
\> Program in Media Arts and Sciences \\ 
\end{tabbing}


%%% Local Variables:
%%% mode: latex
%%% TeX-master: "CharlesHolbrow_MAS_Thesis"
%%% End:
