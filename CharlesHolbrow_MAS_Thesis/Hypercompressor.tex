\chapter{The Hypercompressor}
\label{ch:hypercompressor}

The idea for Hypercompression came during the development of Vocal
Vibrations.\cite{Holbrow2014} What if we could position audio in space
as carefully and maticulously as we can position audio in time?

\section{Ambisonics}
\label{sec:ambisonics}
Ambisonic audio is a technique for encoding and decoding three-dimensional
surround sound audio.\cite{Gerzon1985} Ambisonics differs from other
surround sound formats like $5.1$ and $7.1$ in that it does not depend
on a particular speaker configuration. An ambisonic recording can be
decoded on any surround sound speaker configuration without
disarranging the spatial contents of the audio recording.

Immagine we used an omnidirectinoal microphone to record an acoustic
instrument at a sample rate of $44.1$ kHz. We sample and record 44100
samples every second that represent the air pressure at the microphone
capsule during the recording. Our omni-directional microphone is
designed to treat sound arriving from all angles equally. The
omnidirectional microphone sums together sounds arriving from all
angles and the acoustic directional information is lost.

If we want to encode, decode, transmit, or play audio that preserves
full sphere 360 degree information, ambisonics offers a solution.
Ambisonic audio uses \textit{spherical harmonics} to encode surround
sound audio that preserves the direction-of-arrival information that
discrete channel recordings (such as mono and stereo) cannot fully
capture.

\subsection{Spherical Harmonics}
\label{sec:spherical-harmonics}
We know that we can construct any monophonic audio waveform by summing
a (possibly infinite) number harmonic sine waves (fourier
series).\sidenote{An excelent description of the transormation between
  the time domain and frequency domain can be fount at
  \url{http://betterexplained.com/articles/an-interactive-guide-to-the-fourier-transform/}}
For example, by summing odd \textit{order} sine harmonics at a given
frequency $f$, $(1f, 3f, 5f, 7f, \ldots )$, we generate a square wave with
fundamental frequency $f$. As the order increases, so does the
temporal resolution of our square wave.

By summing sinusoidal harmonics, we can generate any continuous
waveform defined over one dimension. Similarly, by summing
\emph{spherical harmonics}, we can generate any continous shape
defined over the surface of a three-dimensional sphere. Where a
traditional monophonic audio encoding might save one sample 44100
times per second, an ambisonic encoding would save one sample
\emph{for each spherical harmonic} 44100 times per second. This way we
capture a three-dimensional sound image at each audio sample.  The
number of spherical harmonics we encode is determined by our
\textit{ambisonic order}. As our ambisonic order increases, so does
the angular resolution of our result on the surface of the sphere.

\subsection{Spherical Harmonic Definition}
For encoding and decoding ambisonics, the convention is to use the
real portion of spherical harmonics as defined in
equation~\ref{eq:spherical}, where:
\begin{itemize}
\item $Y_{n}^{m}(\varphi,\vartheta)$ is a spherical harmonic that
is:\sidenote{Some literature on spherical harmonics swaps the names
  for \textit{order} and \textit{degree}. In this thesis we use
  $Y_{order}^{degree}$. In literature where $Y_{degree}^{order}$ is
  used, the functions of the subscript and superscript remain
  unchanged; only the names are inconsistent.}
\begin{itemize}
\item of order, $n$
\item of degree, $m$
\item defined over spherical coordinates $(\varphi, \vartheta)$
\end{itemize}
\item $N_n^{|m|}$ is a normalization factor.\sidenote{In ambisonic
    literature (and software), there are multiple incompatible
    conventions for the normalization of spherical harmonics. This
    thesis asumes the SN3D convention as recomended by the AMBIX
    format specification.}\cite{Nachbar2011}
\item $P_n^{|m|}$ is the associated Legendre function of order $n$,
  and degree $m$.
\end{itemize}
\begin{equation}
Y_{n}^{m}(\varphi,\vartheta)=N_n^{|m|}P_n^{|m|}(\sin{\vartheta})
\begin{cases}\label{eq:spherical}
\sin{|m|\varphi},&  \text{for $m<0$}\\  
\cos{|m|\varphi},& \text{for $m\geq 0$}\\
\end{cases}
\end{equation}
Given equation~\ref{eq:spherical}, we can define an ambisonic
audio recording as:
\begin{equation}
f(\varphi,\vartheta,t)=\sum\limits_{n=0}^N\sum\limits_{m=-n}^nY_n^m(\varphi,\vartheta)\phi_{nm}(t)
\label{eq:ambisonics}
\end{equation}
Where:
\begin{itemize}
\item $\varphi$ and $\vartheta$ describe the polar angle of sound
  arrival in two dimensions.\sidenote{Note that ambisonics uses polar
    angles to describe the angle of arrival of sound. These are
    similar to spherical coordinates, minus the inclusion of
    \textit{radial distance}. Distance is not part of the ambisonic
    specification.}
\item $t$ is time
\item $\phi_{nm}(t)$ are our \textit{expansion coeficients}, described
  below.
\end{itemize}

\subsection{Spherical Harmonic Expansion Coeficients}
\label{sec:spher-harm-expans}
In our monophonic recording example, we save just one sample 44100
times per second, representing the fluctuations in air pressure. We
know that by summing the correct combination of spherical harmonics,
we can describe any continous function over the surface of a
sphere. Instead of sampling air pressure directly, we sample a
coefficient describing the weighting of each spherical harmonic 44100
time per second. The resulting sphere corresponds to the pressure of
sound ariving from all directions. The weighting coefficients or
\textit{expansion coefficients} are recorded in our audio file instead
of values representing air pressure directly. Now, by summing together
our weighted spherical harmonics, we can reconstruct the fluctuations
in pressure including the angle or arrival at our sample rate of 44100~kHz.

\subsection{Spatial Resolution}
\label{sec:spatial-resolution}



If insead of 
of recording a sample that represents fluctuations in air pressure, we
record several samples, each representing the weight of a
corresponding spherical harmonic, we can evaluate the 


, we record one sample for each spherical harmonic, we
can reconstruct the full sphere surround sound content at the audio
rate. This process is known as expansion, and and the expansion
coeficients

In the previous section, we saw how increasing the order of 


%%% Local Variables:
%%% mode: latex
%%% TeX-master: "CharlesHolbrow_MAS_Thesis"
%%% End:
