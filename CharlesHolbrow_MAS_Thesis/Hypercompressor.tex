\chapter{The Hypercompressor}
\label{ch:hypercompressor}

The idea for Hypercompression came during the development of Vocal
Vibrations.\cite{Holbrow2014}

\section{Ambisonics}
\label{sec:ambisonics}
Ambisonics is a technique for encoding and decoding three-dimensional
surround sound audio.\cite{Gerzon1985} Ambisonics differs from other
surround sound formats like $5.1$ and $7.1$ in that it does not depend
on a particular speaker configuration. An ambisonic recording can be
decoded on any surround sound speaker configuration without
disarranging the spatial contents of the audio recording. Ambisonic recordings
use \textit{spherical harmonics} to encode three-dimensional surround
sound audio.

\subsection{Spherical Harmonics}
\label{sec:spherical-harmonics}
We know that we can construct any monophonic audio waveform by summing
a (possibly infinite) number harmonic sine waves (fourier
series).\sidenote{An excelent description of the transormation between
  the time domain and frequency domain can be fount at
  \url{http://betterexplained.com/articles/an-interactive-guide-to-the-fourier-transform/}}
For example, by summing odd \textit{order} sine harmonics at a given
frequency $f$, $(1f, 3f, 5f, 7f, \ldots )$, we generate a square wave with
fundamental frequency $f$. As the order increases, so does the
temporal resolution of our square wave.

By summing sinusoidal harmonics, we can generate any continuous
waveform defined over one dimension. Similarly, by summing
\emph{spherical harmonics}, we can generate any continous shape
defined over the surface of a three-dimensional sphere. Where
traditional monophonic audio encoding might save one sample 44100
times per second, an ambisonic encoding would save one sample
\emph{for each spherical harmonic} 44100 times per second. This way we
capture a three-dimensional sound image at each audio sample.  The
number of spherical harmonics we encode is determined be our
\textit{ambisonic order}. As our ambisonic order increases, so does
the angular resolution of our result on the surface of the sphere.

\subsection{Laplace's Spherical Harmonics}
For ambisonics, the convention is to use the real portion of Laplace's
spherical harmonics to encode three-dimensional audio. 
\begin{equation}
Y_{n}^{m}(\varphi,\vartheta)=N_n^{|m|}P_n^{|m|}(\sin{\vartheta})
\begin{cases}
\sin{|m|\varphi},&  \text{for $m<0$}\\  
\cos{|m|\varphi},& \text{for $m\geq 0$}\\ 
\end{cases}
\end{equation}


Each spherical harmonic is described as the output of a
function with two arguments, \textit{order} ($n$), and
\textit{degree} ($m$).

\[
Y_{n=0}^{m=0}()=\frac{1}{2 \sqrt{\pi }}
\]

\[\frac{1}{2 \sqrt{\pi }} \]



While the fourier series is quite simple $(\sin(1f), \sin(2f),
\sin(3f), \ldots)$, spherical harmonics become increasingle complex as
the order increases. 

The derivation of spherical harmonics is described
elsewhere.\cite{Williams1999,Zotter2009a}


generating spherical harmonics is more involved,\TODO{cite, links in
  Kronlacher's spatial transforms for the alteration of ambisonic recordings}.



%%% Local Variables:
%%% mode: latex
%%% TeX-master: "CharlesHolbrow_MAS_Thesis"
%%% End:
