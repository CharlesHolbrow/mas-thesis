
\chapter{The Hypercompressor}
\label{ch:hypercompressor}

\section{Ambisonics}
\label{sec:ambisonics}

\subsection{Spherical Harmonics}
\label{sec:spherical-harmonics}

We know that we can construct any band limited signal by summing a
finite number harmonic sine waves.\sidenote{The best explaination
  I've found of how a function may be described in the frequency
  domain or the time domain can be found at \TODO{link better
    explained site}} For example, by summing even order sine harmonics
at a given frequency f (f, 2f, 4f, 6f, etc) , we generate a square
wave with fundamental frequency f. As the order increases, so does the
resolution of our square wave.

By summing sine wave harmonics, we can generate any continuous
function defined over one dimension. Similarly, by summing \emph{Spherical
  Harmonics}, we can generate any function defined over the surface of
a sphere. As the order of harmonic increases, the so does the
resolution of our output on the surface of the sphere. While it's
quite simple to generate sine harmonics (1f, 2f, 3f, 4f...),
generating spherical harmonics is more involved\TODO{cite, links in
  Kronlacher's spatial transforms for the alteration of ambisonic recordings}.



%%% Local Variables:
%%% mode: latex
%%% TeX-master: "CharlesHolbrow_MAS_Thesis"
%%% End:
