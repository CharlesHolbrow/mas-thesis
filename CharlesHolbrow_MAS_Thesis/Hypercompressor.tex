\chapter{The Hypercompressor}
\label{ch:hypercompressor}

The idea for Hypercompression came during the development of Vocal
Vibrations.\cite{Holbrow2014}

\section{Ambisonics}
\label{sec:ambisonics}
Ambisonics is a technique for encoding and decoding three-dimensional
surround sound audio.\cite{Gerzon1985} Ambisonics differs from other
surround sound formats like $5.1$ and $7.1$ in that it does not depend
on a particular speaker configuration. An ambisonic recording can be
decoded on any surround sound speaker configuration without
disarranging the spatial contents of the audio recording. Ambisonic recordings
use \textit{spherical harmonics} to encode three-dimensional surround
sound audio.

\subsection{Spherical Harmonics}
\label{sec:spherical-harmonics}
We know that we can construct any monophonic audio waveform by summing
a (possibly infinite) number harmonic sine waves (fourier
series).\sidenote{An excelent description of the transormation between
  the time domain and frequency domain can be fount at
  \url{http://betterexplained.com/articles/an-interactive-guide-to-the-fourier-transform/}}
For example, by summing odd \textit{order} sine harmonics at a given
frequency $f$, $(1f, 3f, 5f, 7f, \ldots )$, we generate a square wave with
fundamental frequency $f$. As the order increases, so does the
temporal resolution of our square wave.

By summing sinusoidal harmonics, we can generate any continuous
waveform defined over one dimension. Similarly, by summing
\emph{spherical harmonics}, we can generate any continous shape
defined over the surface of a three-dimensional sphere. Where a
traditional monophonic audio encoding might save one sample 44100
times per second, an ambisonic encoding would save one sample
\emph{for each spherical harmonic} 44100 times per second. This way we
capture a three-dimensional sound image at each audio sample.  The
number of spherical harmonics we encode is determined be our
\textit{ambisonic order}. As our ambisonic order increases, so does
the angular resolution of our result on the surface of the sphere.

\subsection{Spherical Harmonic Definition}
For encoding and decoding ambisonics, the convention is to use the
real portion of spherical harmonics as defined in
equation~\ref{eq:spherical}, where:
\begin{itemize}
\item $Y_{n}^{m}(\varphi,\vartheta)$ is a spherical harmonic that
is:\sidenote{Some literature on spherical harmonics swaps the names
  for \textit{order} and \textit{degree}. In this thesis we use
  $Y_{order}^{degree}$. In literature where $Y_{degree}^{order}$ is
  used, the function of the subscript and superscript remains
  unchanged; only the names are inconsistent.}
\begin{itemize}
\item of order, $n$
\item of degree, $m$
\item defined over spherical coordinates $(\varphi, \vartheta)$
\end{itemize}
\item $N_n^{|m|}$ is normalization factor.\sidenote{In ambisonic
    literature (and software), there are multiple incompatible
    conventions for the normalization of spherical harmonics. This
    thesis asumes the SN3D convention as recomended by the AMBIX
    format specification.}\cite{Nachbar2011}
\item $P_n^{|m|}$ is the associated Legendre function of order $n$,
  and degree $m$.
\end{itemize}
\begin{equation}
Y_{n}^{m}(\varphi,\vartheta)=N_n^{|m|}P_n^{|m|}(\sin{\vartheta})
\begin{cases}\label{eq:spherical}
\sin{|m|\varphi},&  \text{for $m<0$}\\  
\cos{|m|\varphi},& \text{for $m\geq 0$}\\
\end{cases}
\end{equation}
Given equation~\ref{eq:spherical}, we can define an ambisonic
audio recording as:
\begin{equation}
f(\varphi,\vartheta,t)=\sum\limits_{n=0}^N\sum\limits_{m=-n}^nY_n^m(\varphi,\vartheta)\phi_{nm}(t)
\label{eq:ambisonics}
\end{equation}
where:
\begin{itemize}
\item $\varphi$ and $\vartheta$ are the angle of sound arrival in
  spherical coordinates
\item $t$ is time
\item $\phi_{nm}(t)$ are the 
\end{itemize}
Note that while ambisonics describe the angles using the spherical
coordinate convention, distance is not part of the specification.


%%% Local Variables:
%%% mode: latex
%%% TeX-master: "CharlesHolbrow_MAS_Thesis"
%%% End:
