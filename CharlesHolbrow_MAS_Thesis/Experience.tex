\clearpage
\chapter{\textit{De L'Exp\'{e}rience}}
\label{ch:of-experience}

\textit{De L'Exp\'{e}rience} is a composition by Tod Machover for
narrator, organ, and electronics. The piece was commissioned by the
Orchestre Symphonique de Montr\`{e}al (OSM), and premiered at the
Maison Symphonique de Montr\'{e}al on May 16th, 2015. 



Performers include Jean-Willy Kunz, organist in residence with the
OSM, and narrator Gilles Renaud. A recording of the perrmance is
available
online.\sidenote{\url{http://web.media.mit.edu/~holbrow/mas/TodMachover_OfExperience_Premier.wav}}

The text for the piece was taken from the writings of
Michel de Montaigne. The

\subsection{The Organ}
\label{sec:organ}
The project presented a unique challenge. The goal is to blend the
acoustic sound of the organ with with the 


The pipe organ is perhaps the best example of an instrument that
occupies space.

While pipe organs do come in many shapes and sizes,
the Pierre B\'{e}ique Organ in the Maison Symphonique de Montr\'{e}al
is amo
The concert hall at the OSM features a very impressive 
6489


\subsection{Sound Reinforcement}
\label{sec:sound-reinforcement}
Loudspeakers were positioned throughout the hall. A number of factors
went into the arrangement: audience coverage, surround coverage,
rigging availability, and setup convenience. All speakers used were by
Myer Sound.\sidenote{\url{http://www.meyersound.com/}} A single CQ-2
was positioned just behind and above the narrator, to help localize
the image of his voice.  JM-1P speakers on stage left and and stage
right were also used for the voice of the narrator, and incorporated
into the ambisoic playback system. Ten pairs of UPJ-1Ps were placed in
the hall, filling in the sides and rear for ambisonic playback, Two at
the back of the hall, mirroring the CQ-2s on stage, four on each of
the first and third balconies.  The hall features variable acoustics,
and curtains can be drawn into the hall to increase acoustic
absorbtion, and decrease reverb time. These were partially engaged,
striking a balance: The reduced reverb time improved the clarity of
the amplified voice, while only marginally impacting the beautiful
acoustic deacay of the organ in the hall. The show was mixed by Ben
Bloomberg. Ambisonic playback and multitrack recording of the
performance was made possible with the help and expertiese of Fabrice
Boissin and Julien Boissinot and the Centre for Interdisciplinary
Research in Music Media and Technology (CIRMMT) at McGill University.







%%% Local Variables:
%%% mode: latex
%%% TeX-master: "CharlesHolbrow_MAS_Thesis"
%%% End:
