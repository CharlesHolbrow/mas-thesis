\clearpage
\chapter{\textit{De L'Exp\'{e}rience}}
\label{ch:experience}

\textit{De L'Exp\'{e}rience} is a composition by Tod Machover in eight
sections for narrator, organ, and electronics. The piece was
commissioned by the \textit{Orchestre Symphonique de Montr\`{e}al}
(OSM) and premiered at the \textit{Maison Symphonique de Montr\'{e}al}
on May 16th, 2015. The text for the piece was taken from the writings
of Michel de Montaigne, the 16th century philosopher known for
popularizing the essay form.  Performers included Jean-Willy Kunz,
organist in residence with the OSM, and narrator Gilles Renaud. A
recording of the performance has been made available
online.\sidenote{\url{http://web.media.mit.edu/~holbrow/mas/TodMachover_OfExperience_Premier.wav}}


\subsection{The Organ}
\label{sec:organ}
The live performance of \textit{De L'Exp\'{e}rience} presented a
unique challenge that fits well with the themes in this thesis. The
acoustic pipe organ can project sound into space unlike any array of
loudspeakers. This is especially true for an instrument as large and
magnificent as the Pierre B\'{e}ique Organ in the OSM concert hall,
which has 6489 pipes and extends to approximately 10 meters above the
stage. Our objective is to blend the sound of the organ with the sound
of electronics.
% The design of the organ is a collaboration between
% Diamond Schmitt Architects and Quebec-based organ manufacturer
% Casavant.

\section{Electronics}
\label{sec:electronics}
The electronics in the piece included a mix of synthesizers,
pre-recorded acoustic cello, and other processed material from
acoustic and electronic sources, all composed by Tod Machover. Prior
to the performance, these sounds were mixed ambisonically:
\begin{enumerate}
\item The cello was placed in front, occupying approximately the front
  hemisphere of our surround sound image.
\item The left and right channel of the electronic swells were panned
  to the left and right hemispheres. However, by default they were
  collapsed to omnidirectional mono (the sound comes from all
  directions, but has no stereo image). The gain of this synth was
  mapped to directionality, so when the synth grows louder, the left
  and right hemisphere become distinct from each other, creating
  an illusion of the sound is growing larger.
\item Additional sound sources are positioned in space, such that each
  has as wide an image as possible, but overlaps with others as little
  as possible.
\end{enumerate}
The overarching goal of this approach was to create a diverse but
interesting spatial arrangement, while keeping sounds mostly panned in
the same spot: movement comes from the warping of the surround image
by the Hypercompressor.

\subsection{Sound Reinforcement}
\label{sec:sound-reinforcement}
Loudspeakers were positioned throughout OSM concert the hall. A number
of factors went into the arrangement: audience coverage, surround
coverage, rigging availability, and setup convenience. All speakers
used were by Meyer Sound.\sidenote{\url{http://www.meyersound.com/}} A
single CQ-2 was positioned just behind and above the narrator to help
localize the image of his voice.  JM-1P speakers on stage left and
stage right were also used for the voice of the narrator, and
incorporated into the ambisonic playback system. Ten pairs of UPJ-1Ps
were placed in the hall, filling in the sides and rear for ambisonic
playback, two at the back of the hall, mirroring the CQ-2s on stage,
four on each of the first and third balconies.  The hall features
variable acoustics, and curtains can be drawn into the hall to
increase acoustic absorption and decrease reverb time. These were
partially engaged, striking a balance: The reduced reverb time
improved the clarity of amplified voice, while only marginally
impacting the beautiful acoustic decay of the organ in the hall. The
show was mixed by Ben Bloomberg. Ambisonic playback and multitrack
recording of the performance was made possible with the help and
expertise of Fabrice Boissin and Julien Boissinot and the Centre for
Interdisciplinary Research in Music Media and Technology (CIRMMT) at
McGill University.

\paragraph{A note on composition, performance, and engineering}
No amount of engineering can compensate for poor composition,
orchestration, or performance. A skilled engineer with the right tools
only can only mitigate shortcomings in a performance. Good engineering
starts and ends with good composition, arrangement and performance. I
have been quite fortunate that all the musicians involved with
\textit{De L'Exp\'{e}rience} at every stage are of the highest
caliber.

\section{Live Hypercompression Technique}
\label{sec:live-hyperc-techn}
During the performance, the encoded ambisonic electronic textures were
patched into the main input of the Hypercompressor, before being
decoded in realtime using the \textit{Rapture3D Advanced} ambisonic
decoder by Blue Ripple
Sound.\sidenote{\url{http://www.blueripplesound.com/products/rapture-3d-advanced}}
Four microphones captured the sound of the organ: two inside and two
hanging in front. The placement of the mics was intended to capture as
much of the sound of the organ as possible, and as little of the sound
of the amplified electronics in the hall. These four microphone
signals where encoded to ambisonics in realtime, and the resulting
ambisonic feed was patched into the side-chain input of the
Hypercompressor. In this configuration, the organ drives the
spatialization of the electronic sounds. By ambisonically panning the
organ microphones, we can control how our electronics are
spatialized. After some experimentation we discovered the best way to
apply the Hypercompressor in the context of \textit{De
  L'Exp\'{e}rience}. When the organ was played softly, the sound of
the electronics filled the performance hall from all directions. As
the organ played louder, the electronic textures dynamically warped
toward the organ in the front of the concert hall. The spatial and
timbral movement of the electronics together with the magnificent (but
stable) sound of the the organ created a unique blend that would be
inaccessible with acoustic or electronic sounds in isolation.

\begin{figure*}[]
  \includegraphics[width=\linewidth]{DressRehersal.jpg}
  \caption{The Pierre B\'{e}ique Organ in the OSM concert hall during
    a rehearsal on May 15th, 2015. Approximately 97\% of the organs'
    6489 pipes are out of sight behind the woodwork. Photo credit: Ben
    Bloomberg}
  \label{fig:le-corbusier-sketch}
\end{figure*}


%%% Local Variables:
%%% mode: latex
%%% TeX-master: "CharlesHolbrow_MAS_Thesis"
%%% End:
