\clearpage
\chapter{Discussion and Analysis}
\label{ch:analysis}
In the previous chapters, we explore three new tools for creating and
processing music, their motivations, and implementations. The \refmod
in chapter \ref{ch:ref-mod}, proposes a new way to think about sound
in space. The \polytempic in chapter \ref{ch:polytempic} does the same
with musical time and meter. Chapters \ref{ch:hypercompressor} and
\ref{ch:experience} describe and implement a technique for modulating
music in in space with time as a reference. Each project project
builds on Iannis Xenakis' theory of \textit{stochastic music}. Each
project incorporates elements from other disciplines including
mathematics, computer science, acoustics, audio engineering and
mixing, sound reinforcement, multimedia production, and live
performance.

This final chapter discusses how each project succeeded, how each
project failed, and how future iterations can benefit from lessons
learned during the development process. 

\section{Evaluation Criteria}
\label{sec:eval-criteria}
To evaluate a project of any kind, it is helpful to begin with a
purpose, and then determine the granularity and scope of the
evaluation.\cite{Saltzer2009} We might evaluate a music recording for
audio fidelity, for musical proficiency of the artist, for emotional
impact or resonance, for narrative, for technological inovation, for
creative vision, or for political and historical insight. This also
applies to the evaluation of technology.  We can evaluate the
suitability of a rack-mount analog to digital converter (ADC) for a
given purpose. A recording engineer may prefer the device impart a
favorable sound, while an acoustician may prefer that the device be
as neutral as possible. However, when evaluating an ADC or a music
recording, what we are most concerned with is the top level interface:
From most perspectives, we evaluate music by listening to it, and we
are not concerned which ADC was used to make the recording. When an
audio engineer, or acoustician evaluates an ADC, the device's
performance is more important than the exact layout of electronic
components inside the device.

Stochastic music theory is a \textit{vertical integration} of
mathematics, the physics of sound, psychoacoustics, and music. The
theory of stochastic music begins with the lowest level components of
sound, and ends with a creative musical product. What is a reasonable
perspective from which to evaluate stochastic music? From the
perspective of a listening, or performing the music? From the
perspective of a historian, evaluating the environment that led to the
composition, or studying the impact on music afterwords?  Should we try
to make sense of the entire technology stack, or try to evaluate every
layer of abstraction individually?

Somehow, between the low-level elements of sound and a musical
composition or performance, we transition from what is quantifiable to
what we can only attempt to describe. This impossible challenge of
quantifying the unquantifiable is exactly what makes music technology
and audio engineering so appealing.


%  Does the unit have an appropriate number and type of audio outputs suit
% our needs (such as TRS, XLR, and USB)? Does it perform reliably? Does
% it have a neutral or favorable impact on the sound? The results of our
% evaluation will depend on our purpose. If the purpose of the device is
% for playback of audio in a live performance context, the best choice
% will be different than if we want to use the unit for mixing in a
% recording studio.  

\section{\refmod}

\section{\polytempic}
Even just explaining the problem has proven to be tremendously
difficult. 

\section{\thesis}
welp, not sure how well it actually
worked. And I don't expect electronic instrument to ever replace
acoustic instruments, although they may eventually equal them in
expressivity. Initially thought it would work backwards

The design and development of the hypercompressor happened in
parallel, and many design decisions where meant for this one project,
not the most general cases. 
%%% Local Variables:
%%% mode: latex
%%% TeX-master: "CharlesHolbrow_MAS_Thesis"
%%% End:
