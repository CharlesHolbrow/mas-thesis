\clearpage
\chapter{Discussion and Analysis}
\label{ch:analysis}
In the previous chapters, we explore three new tools for creating and
processing music, their motivations, and implementations. The \refmod
in chapter \ref{ch:ref-mod}, proposes a new way to think about sound
in space. The \polytempic in chapter \ref{ch:polytempic} does the same
with musical time and meter. Chapters \ref{ch:hypercompressor} and
\ref{ch:experience} describe and implement a technique for modulating
music in in space with time as a reference. Each project project
builds on Iannis Xenakis' theory of \textit{stochastic music}. Each
project incorporates elements from other disciplines including
mathematics, computer science, acoustics, audio engineering and
mixing, sound reinforcement, multimedia production, and live
performance.

This final chapter discusses how each project succeeded, how each
project failed, and how future iterations can benefit from lessons
learned during the development process.

\section{Evaluation Criteria}
\label{sec:eval-criteria}
To evaluate a project of any kind, it is helpful to begin with a
purpose, and then determine the granularity and scope of the
evaluation.\cite{Saltzer2009} We might evaluate a music recording for
audio fidelity, for musical proficiency of the artist, for emotional
impact or resonance, for narrative, for technological inovation, for
creative vision, or for political and historical insight. This also
applies to the evaluation of technology.  We can evaluate the
suitability of a rack-mount analog to digital converter (ADC) for a
given purpose. A recording engineer may prefer the device impart a
favorable sound, while an acoustician may prefer that the device be
as neutral as possible. However, when evaluating an ADC or a music
recording, what we are most concerned with is the top level interface:
From most perspectives, we evaluate music by listening to it, and we
are not concerned which ADC was used to make the recording. When an
audio engineer, or acoustician evaluates an ADC, the device's
performance is more important than the exact layout of electronic
components inside the device.

Stochastic music theory is a \textit{vertical integration} of
mathematics, the physics of sound, psychoacoustics, and music. The
theory of stochastic music begins with the lowest level components of
sound, and ends with a creative musical product. What is a reasonable
perspective from which to evaluate stochastic music? From the
perspective of a listening, or performing the music? From the
perspective of a historian, evaluating the environment that led to the
composition, or studying the impact on music afterwords?  Should we try
to make sense of the entire technology stack, or try to evaluate every
layer of abstraction individually?

Somehow, between the low-level elements of sound and a musical
composition or performance, we transition from what is quantifiable to
what we can only attempt to describe. This impossible challenge of
quantifying the unquantifiable is exactly what makes music technology
and audio engineering so appealing.

\TODO{What approach did I use?}

%  Does the unit have an appropriate number and type of audio outputs suit
% our needs (such as TRS, XLR, and USB)? Does it perform reliably? Does
% it have a neutral or favorable impact on the sound? The results of our
% evaluation will depend on our purpose. If the purpose of the device is
% for playback of audio in a live performance context, the best choice
% will be different than if we want to use the unit for mixing in a
% recording studio.

\section{\refmod}
This project provides a single abstract interface that approaches
composition of space (architecture) and the composition of music at
the same time. The forms that it makes are familiar from the ruled
surfaces seen in Xenakis compositions, and early sketches of the
Philips Pavilion. In musical mode, we can think of the x and y axes
representing time and pitch. In architectural mode the canvas might
represent the floor plan of spaces we are designing.

While it is interesting to switch our perspective between the two
modes, there is not a clear connection from one to the other. A
carefully designed surface or reflection in one mode would be quite
arbitrary in the other mode. The reason that the interface is capable
of working in both modes is because it is so minimalist, that it does
not commit to one or the other. This is not a complete failing: The
tool was really designed to be a brainstorming aid at the very
beginning of the design process. It can be much simpler and quicker to
use that proper architectural software, as a means of creating
abstract shapes, similar to sketching on paper, before turning to
specialized software for more detailed design.

\paragraph{Curves, Constraints, and Simplicity} Despite
the shortcomings in this project, the parts that worked well and make
a good starting point for future iterations. There's something
intangible, but simple and \textit{fun} about the user
interface. There is only one input action; dragging a control
point. It's immediately clear what each control point does, and it is
easy to not even notice that there are five different types of control
points and each has slightly different behavior. It is very intuitive
to adjust a reflection surface such that the red beams \textit{focus}
on a certain point, and then re-adjust a reflection surface so that
they diverge chaotically. There is something fascinating about how the
simple movements cause simultaneously coordinated and chaotic
results. The response might be described as \textit{stochastic}!

The red ``sound lines'' have three degrees of freedom: Position,
direction, and length. We can point the rays in any direction we like,
but, their movement is somewhat constrained.  The projection angle is
locked to 30 degrees, and the number of beams is always 8. Most of the
flexibility from the interface comes from the reflective surfaces.

It is easier to draw a curving reflective surface than a strait
one. If you make a special effort, it is possible to make one of the
surfaces straight, but just like drawing a line on a paper with a pen,
curved surfaces come more naturally. However, the curves in the \refmod do not
come naturally because they are following an input gesture like most
``drawing'' interfaces, but because of the simple mathematics in the
of the Bezier curves. Similarly, if we consider the red lines to be
notes on a time/pitch axis, the default interpretation is stochastic
glissandi rather than static pitches. Most musical software assumes
static pitches by default, and most architectural software assumes
strait lines by default.

\paragraph{Next Steps}
The obvious next steps, for this project, are correcting the
shortcomings described above. It could be made to work in three
dimensions, and model precise propagation of sound, rather than a very
simplified abstraction: It could become a proper acoustical
simulator. Another possibility is playing the sound is turning it into
a musical instrument where we can hear the stochastic glissandi in
realtime. These options are not necessarily mutually exclusive, but as
the interface becomes tailored to a more specific application, our
ability to think about the content as abstract representations also
breaks down. The ideal of software that is equally well equipped to
compose music and to imagine architectural spaces is probably
unrealistic. The beauty of the abstract representation of music
composition, is that \textit{any} visual representation of music is
quite abstract.

% Describe where I want to take this.
\section{\polytempic}

\paragraph{Next Steps}

Even just explaining the problem has proven to be tremendously
difficult. 

\section{\thesis}

\paragraph{Next Steps}
welp, not sure how well it actually
worked. And I don't expect electronic instrument to ever replace
acoustic instruments, although they may eventually equal them in
expressivity. Initially thought it would work backwards

The design and development of the hypercompressor happened in
parallel, and many design decisions where meant for this one project,
not the most general cases. 
%%% Local Variables:
%%% mode: latex
%%% TeX-master: "CharlesHolbrow_MAS_Thesis"
%%% End:
