\chapter{Time Domain: \polytempic}
\label{ch:polytempic}
In the 20th century there was an revival of complexity in contemporary
music composition.\sidenote[][-5mm]{Works by Italian composer Luciano
  Berio illustrate the complexity of post-war contemporary
  music. Beginning in 1958, Berio wrote a series of works he called
  \textit{Sequenza}. Each was a highly of highly technical composition
  written for a virtuosic soloists. Each was for a different
  instrument ranging from flute to guitar to accordion.  In
  Sequenza~IV, for piano, Berio juxtaposes thirty-second note
  quintuplets, sextuplets, and septuplets (each with a different
  dynamic), over just a few measures.} As performers developed the
virtuosic skills required to play the music, composers also wrote
increasingly difficult scores to challenge the
performers.\cite{grout2006} One composer writing very complex music in
this period was Elliot Carter. Carter developed a technique he called
\textit{metric modulation}, in which his music would transition from
one musical meter to another through a transitional section that
shared aspects of both the original meter and the destination
meter.

While metric modulation is a technique for changing meter, and
and \polytempic is a technique for changing tempo, the former led to
the later in a surprising way. Carter's reputation for complexity in
music attracted the attention of composer and cellist Tod
Machover. While Machover was studying with Carter, wrote a trio for
violin, viola, and cello in which each instrument would accelerate or
decelerate relative to the others. The piece was so difficult, that
it was impossible to find anyone who could play it correctly. In a
recent interview, Machover recalls the experience:
\begin{quotation}``A sort of lightbulb went off\ldots{} computers are out
  there, and if you have an idea and can learn how to program, you
  should be able to model it.''\cite{Fein2014}
\end{quotation}
We can see how this insight led Machover to begin programming
computers to make sound, to IRCAM in Paris, and Finally to the MIT
Media Lab where the \polytempic was developed. 

\section{Stochos}
\label{sec:polytempic-stochos}
In \autoref{ch:introduction} (see figure~\ref{fig:metastasis}) we saw
how Xenakis was using ruled surfaces to create swarms of notes that
move together, creating stochastic sonorities. The goal of \polytempic
is to enable composition with swarms of tempo modulations that move in
correlated, cohesive patterns. Music with two or more simultaneous
tempos (polytempic music) is itself not a new concept, and many
examples of polytempic music exist.\cite{Greschak2003} Slightly less
common is polytempic music where continuous tempo accelerations or
decelerations are defined relative to each other. This style of music
is well suited to tape music, because tape machines can play
recordings back at variable rates. However, it is difficult to control
the exact point (or phase) when de-synchronized tape becomes
re-aligned. Performative music with simultaneous tempi that accelerate
and decelerate relative to each other is unusual, but does exist. In
a 1971 interview composer Steve Reich described how he made the
transition to performative polytempic music after working on his tape
music composition, \textit{Come Out}:
\begin{quotation}
  ``1966 was a very depressing year. I began to feel like a mad
  scientist trapped in a lab: I had discovered the phasing process
  of Come Out and didn't want to turn my back on it, yet I didn't know
  how to do it live, and I was aching to do some instrumental
  music. The way out of the impasse came by just running a tape loop
  of a piano figure and playing the piano against it to see if in fact
  I could do it. I found that I could, not with the perfection of the
  tape recorder, but the imperfections seemed to me to be interesting
  and I sensed that they might be interesting to listen to.''\cite{Nyman2015}
\end{quotation}
Reich's experience illustrates what other composers and performers
have also encountered: It is quite difficult to perform polytempic
music accurately. In \textit{Piano Phase} Reich has two performers
playing the same 12 tone series on the piano. After a set number of
repetitions through the pattern, one performer begins to play slightly
faster until she is exactly one note ahead of the other performer, at
which point both performers play at the same rate for a time. This
process is repeated and iterated on, creating a live \emph{phasing}
effect without the pitch shifting that would occur when phasing analog
tape. If we compare a live performance\cite{Huisman1989} with a
programatic rendering\cite{Chen2014} of \textit{Piano Phase}, we can
hear how the programatic rendering is able to accelerate more
smoothly. The programatic example spends longer on the transitions
where the two parts are out of phase.

\section{Objective}
\label{sec:polytempic-objective}
Steve Reich composed \textit{Piano Phase} for two performers. In his
experimentation, he found that if the music is reasonably simple, two
performers can make synchronized tempo adjustments relative to each
other well enough to yield compelling results. To create stochastic
tempo transitions, our requirements are probably too demanding for
unassisted performers. Our goal is to compose and audition music
where:
\begin{enumerate}
  \item Swarms of an arbitrary number of simultaneous tempi
    coexist. 
  \item Each individual player within the swarm can continuously
    accelerate or decelerate individually, but also as a member of a
    cohesive whole. 
  \item Each musical line can converge and diverge at explicit
    points. At each point of convergence the phase of the meter within
    the tempo can be set.
\end{enumerate}
We start by defining a single tempo transition. Consider the following
example (shown in figure~\ref{fig:basic-tempo-change}):
\begin{itemize}
\item Assume we have 2 snare drum players. Both begin playing the same
  beat at 90 BPM in common time.
\item One performer gradually accelerates relative to the other. We want
  to define a continuous tempo curve such that one drummer accelerates
  to 120 BPM.
\item So far, we can easily accomplish this with a simple linear tempo
  acceleration. However, we want the tempo transition to complete
  exactly when \emph{both} drummers are on a down-beat, so the the
  combined effect is a 3 over 4 rhythmic pattern. \textbf{Linear
    acceleration results in the transition completing at an arbitrary
    phase.}
\item We want the accelerating drummer to reach the new tempo after
  exactly 20 beats.
\item We also want the acceleration to complete in exactly 16 beats of
  the original tempo, so the drummer playing a constant tempo, and the
  the accelerating drummer are playing together.
\end{itemize}
\begin{figure*}[h]
  \includegraphics[width=\linewidth]{basic-tempo-transition.png}
  \caption[Tempo Transition]{Tempo Transition from 90~BPM
    to 120~BPM}
  \label{fig:basic-tempo-change}
\end{figure*}

\section{Solution}
\label{sec:polytempic-solution}
We are interested in both the number of beats elapsed in the static
tempo \emph{and} in the changing tempo, and the absolute tempo. Let us
think of the number of beats elapsed as our \emph{position}, and the
tempo as our \emph{rate}, we see how this resembles a physics
problem. If we have a function that describes our tempo (or rate), we
can integrate that function, and the result will tell us our number of
beats elapsed (or position). Given the above considerations, we define
our tempo curve in terms of 5 constants:
\hfill\break
\begin{fullwidth}
\begin{itemize}
  \item Time $t_0=0$, when the tempo transition begins
  \item A known time, $t_1$, when the tempo transition ends
  \item A known starting tempo, $\dot{x}_0$
  \item A known finishing tempo, $\dot{x}_1$
  \item The number of beats elapsed in the changing tempo between
    $t_0$ and $t_1$, $x_1$
\end{itemize}
\end{fullwidth}
\hfill\break
The tension of the tempo curve determines how many beats elapse during
the transition period. The curve is well-defined for some starting
acceleration $a_0$ and finishing acceleration $a_1$, so we define the
curve in terms of linear acceleration. Using Newtonian notation we can
describe our tempo acceleration as:
\begin{equation}
	\label{accel}
    \ddot{x}_1 = a_0 + a_1t_1
\end{equation}
Integrating linear acceleration (\ref{accel}) yields a quadratic
velocity curve (\ref{bpm}). The velocity curve describes the tempo (in beats per
minute)\marginnote{We must specify the same time units for input
  variables like $t_1$ and $\dot{x_1}$. I prefer \textit{minutes} for
  $t_1$ and \textit{beats per minute} for $\dot{x_1}$ over
  \textit{seconds} and \textit{beats per second}} with respect to
time.
\begin{equation}
	\label{bpm}
    \dot{x}_1 = \dot{x}_0 + a_0t_1 + \frac{a_1t_1^2}{2}
\end{equation}
Integrating velocity (\ref{bpm}) gives us a function describing
position (the number of beats elapsed with respect to time).
\begin{equation}
	\label{beats-elapsed}
	x_1 = x_0 + \dot{x}_0t_1 + \frac{a_0t_1^2}{2} + \frac{a_1t_1^3}{6}
\end{equation}
With equations (\ref{bpm}) and (\ref{beats-elapsed}), we can solve for our 
two unknowns, $a_0$ and $a_1$. First we solve both equations for $a_1$:
\begin{displaymath}
    \label{a1-solution}
    a_1=
    \frac{-2}{t_1^2}(\dot{x}_0-\dot{x}_1 + a_0t_1)=
    \frac{-6}{t_1^3}(\dot{x}_0-x_1 + \frac{a_0t_1^2}{2})
\end{displaymath}
Assuming $t_1 \neq 0$, we solve this system of equations for $a_0$:
\begin{equation}
	\label{a0-result}
	a_0=\frac{6x_1-2t_1(\dot{x}_1+2\dot{x}_0)}{t_1^2}
\end{equation}
Evaluating (\ref{a0-result}) with our constants gives us our starting
acceleration. Once we have $a_0$ we can solve (\ref{bpm}) for $a_1$, and 
evaluate (\ref{bpm}) with $a_1$ and $a_0$ to describe our changing tempo 
with respect to time.

\section{Stochastic Transitions}
\label{sec:polytempic-implementation}
Equipped with out equations from the previous section, it is quite
simple to create swarms of parallel tempos that are correlated and
complex. In figure~\ref{fig:polytempic-transition} we build on the
previous example. Here, each additional tempo curve is calculated the
same way, except $x_1$ (number of beats in our accelerating tempo
during the transition) is incremented for each additional tempo line. 
\begin{figure*}[h!]
  \includegraphics[width=\linewidth]{stochastic-tempi.png}
  \caption{Stochastic Tempo Transition from 90~BPM to 120~BPM. Black
    dots are beats in our changing tempi. Grey dots show a
    continuation of beats at the initial tempo.
    $12 \leq x_1 \leq 20$}
  \label{fig:polytempic-transition}
\end{figure*}\hfill\break
This pattern clearly exhibits controlled chance, and that Xenakis
would describe as \emph{stochastic}. On the very first beat at $t=0$,
all parallel parts are aligned. Beats 2 and 3 can be heard as discrete
rhythmic events, but become increasingly indistinct. The end of beat 4
then overlaps with the start of beat 5, before articulated beats begin
to slide into pseudo random noise. By beat 13 of the static tempo, the
chaos of the many accelerating tempi begin to settle back into order
before returning to complete syncronicity at $t=16$.

\section{Prior Work}
\label{sec:polytempic-prior-work}
Many commercial and research projects deal with different ways to
manipulate rhythm and tempo. Flexible digital audio workstations (DAWs)
like Cockos Reaper\sidenote{\url{http://www.reaper.fm}} and MOTU
Digital
Performer\sidenote{\url{http://www.motu.com/products/software/dp}}
include features for auditioning tracks or music-objects with unique
simultaneous tempi, and individual tempos can even be automated
relative to each other. However, the the precise non-linear tempo
curves that are required for the syncopated musical content to
synchronize correctly after a transition completes, are not possible in
any DAW we tried. Audio programming languages like
Max\sidenote{https://cycling74.com/products/max/} and
SuperCollider\sidenote{http://supercollider.github.io/} could be used
to create tempo swarms, but require equations like the ones defined in
section~\ref{sec:polytempic-solution}. One project, \textit{Realtime
  Representation and Gestural Control of Musical
  Polytempi}\cite{Nash2008} demonstrates an interface for generating
Polytempic music, but is not intended or capable of generating
coordinated or stochastic tempi swarms.  \textit{The Beatbug
  Network}\cite{Weinberg2002} is described as a multi-user interface
for creating stochastic music, but is focused on ``beats'' or musical
rhythmic patterns and timbres, rather than
tempi. \textit{Stochos}\cite{Bokesoy2003} is a software synthesizer
for generating sound using random mathematical distributions, but is
also not designed to work with simultaneous tempos or even work as a
rhythm generator. Finally, \textit{Polytempo Network}\cite{Kocher2014}
is a project that facilitates the performance of polytempic music, but
does not aid the composition thereof.

\section{Future Directions}
\label{sec:polytempic-future-directions}
\polytempic started with a question asked by musician and composer
Bryn Bliska. She wanted compositional constructs like the one in
figure~\ref{fig:basic-tempo-change} in her music, but found that it
was not posible to do accurately with currently available tools. At
the time, neither Bryn or I knew that Tod Machover had initially
turned to computer music to solve exactly this problem.  However, the
equations derived here are only a partial solution. The next step is
to develop an interface for composing and performing with
compositional or performative interface for controlling Stochastic
Tempo Modulations.

\begin{figure*}[]
  \includegraphics[width=\linewidth]{stochastic-tempi-3.png}
  \caption{\polytempic with variable $t_1$ and $x_1$.}
  \label{fig:polytempic-transition-3}
\end{figure*}


%%% Local Variables:
%%% mode: latex
%%% TeX-master: "CharlesHolbrow_MAS_Thesis"
%%% End:
