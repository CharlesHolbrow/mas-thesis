
\chapter{\polytempic}
\label{ch:polytempic}

Eliot Carter: Polymetric Modulation. 
Steve Reich: Piano Phase
Paper: realtime representation of 
Paper: Stochos: Software for Real-Time Synthesis of Stochastic Music

John Cage: Chance Music
Karlheinz Stockhausen, Pierre Boulez, Luciano Berio: Aleatoric music

In figure~\ref{fig:metastasis} we saw how Xenakix was using ruled surfaces to create
swarms of notes that move together to create stochastic
sonorities. Xenakis was among a group of 20th century composers who
were searching for ways to push the boundaries of established music
composition. Other notable composers of the time include John Cage,
who wrote wrote what he called, Chance Music, and  Stockhausen, Boulez
and Berio who wrote aleatoric music. 

The goal of \polytempic music is to enable composition with swarms of
tempo modulations that move in correlated, cohesive patterns. This is
a style that is well suited to tape music, because tape machines can
play recordings back at variable rates. However, it is difficult to
control the exact point when de-synchronized tape becomes
re-aligned. Performative music with simultaneous tempi that accelerate
and decelearate realtive to each other is unusual, but does
exist. In a 1971 interview Composer Steve Reich described how he made
the transition to performative polytempic music after discovering it
while working on his tape music piece, \textit{Come Out}:
\begin{quotation}
  ``1966 was a very depressing year. I began to feel like a mad
  scientist trapped in a lab: I had discovered the phasing process
  of Come Out and didn't want to turn my back on it, yet I didn't know
  how to do it live, and I was aching to do some instrumental
  music. The way out of the impasse came by just running a tape loop
  of a piano figure and playing the piano against it to see if in fact
  I could do it. I found that I could, not with the perfection of the
  tape recorder, but the imperfections seemed to me to be interesting
  and I sensed that they might be interesting to listen to.''\cite{Nyman2015}
\end{quotation}



%%% Local Variables:
%%% mode: latex
%%% TeX-master: "CharlesHolbrow_MAS_Thesis"
%%% End:
