
\chapter{Time Domain Manipulation: \polytempic}
\label{ch:polytempic}
In \autoref{ch:introduction} (figure~\ref{fig:metastasis}) we saw how
Xenakix was using ruled surfaces to create swarms of notes that move
together to create stochastic sonorities. The goal of \polytempic is
to enable composition with swarms of tempo modulations that move in
correlated, cohesive patterns. Music with two or more simultaneous
tempos (polytempic music) is itself not a new concept, and many
examples of polytemic music exist\cite{Greschak2003}. Slightly less
comon is polytempic music where continuous tempo accelerations or
decelerations are defined relative to each other.  This style of music
is well suited to tape music, because tape machines can play
recordings back at variable rates. However, it is difficult to control
the exact point (or phase) when de-synchronized tape becomes
re-aligned. Performative music with simultaneous tempi that accelerate
and decelearate realtive to each other is unusual, but does exist. In
a 1971 interview Composer Steve Reich described how he made the
transition to performative polytempic music after working on his tape
music composition, \textit{Come Out}:
\begin{quotation}
  ``1966 was a very depressing year. I began to feel like a mad
  scientist trapped in a lab: I had discovered the phasing process
  of Come Out and didn't want to turn my back on it, yet I didn't know
  how to do it live, and I was aching to do some instrumental
  music. The way out of the impasse came by just running a tape loop
  of a piano figure and playing the piano against it to see if in fact
  I could do it. I found that I could, not with the perfection of the
  tape recorder, but the imperfections seemed to me to be interesting
  and I sensed that they might be interesting to listen to.''\cite{Nyman2015}
\end{quotation}
Reich's experience illustrates what other composers and performers
have also encountered: It is quite difficult to perform polytempic
music accurately. In \textit{Piano Phase} Reich has two performers
playing the same 12 tone series on the piano. After a set number of
repetitions through the pattern, one performer begins to play slightly
faster until she is exactly one note ahead of the other performer, at
which point both performers play at the same rate for a time. This
process is repeated and iterated on, creating a live \emph{phasing}
effect without the pitch shifting that would occur when phasing analog
tape. If we compare a live performance\cite{Huisman1989} with a
programatic rendering\cite{Chen2014} of \textit{Piano Phase}, we can
hear how the programatic rendering is able to accelearate more
smoothly. The programatic example spends longer on the transitions
where the two parts are out of phase.

\section{Objective}
\label{sec:polytempic-objective}
Steve Reich composed \textit{Piano Phase} for two performers. In his
experimentation, he found that if the music is reasonably simple, two
performers can make synchronized tempo adjustments relative to each
other well enough to yeild compelling results. To create stochastic
tempo transitions, our requirements are probably too demainding for
unassisted performers. Our goal is to compose and audition music
where:
\begin{enumerate}
  \item Swarms of an arbitrary number of simultaneous tempi
    coexist. 
  \item Each individual player within the swarm can continously
    accelerate or decelerate individually, but also as a member of a
    cohesive whole. 
  \item Each musical line can converge and diverge at explicit
    points. At each point of convergence the phase of the meter within
    the tempo can be set.
\end{enumerate}
For example we might have 2 snare drum players. Both begin playing the
same beat at the same tempo, but one player gradually decelerates
relative to the other. We want to define a continous tempo curve such
that one drummer accelerates to four-thirds the original rate. So far,
we can accomplish this with a simple linear tempo acceleration, but we
have an additional constraint. We want the tempo transition to
complete exactly when \emph{both} drummers are on a down-beat. We want
to define the exact number of beats that our drummer uses to complete
the tempo transition, \emph{and} the exact number of beats that elapse
for the first drummer playing a steady beat.

\section{Solution}
\label{sec:polytempic-solution}
Given the above considerations, we define our tempo curve in terms of
5 constants:
\\[5mm]
\begin{fullwidth}
\begin{itemize}
  \item Time $t_0=0$, when the tempo transition begins
  \item A known time, $t_1$, when the tempo transition ends
  \item A known starting tempo, $\dot{x}_0$
  \item A known finishing tempo, $\dot{x}_1$
  \item The number of beats elapsed in the changing tempo between $t_0$ and $t_1$, $x_1$
\end{itemize}
\end{fullwidth}
\\[5mm]
The tension of the tempo curve determines how many beats elapse during 
the transition period. The curve is well-defined for some starting 
acceleration $a_0$ and finishing acceleration $a_1$, so we define the curve 
in terms of linear acceleration. Using Newton's notation we can describe our 
tempo acceleration as:
\begin{equation}
	\label{accel}
    \ddot{x}_1 = a_0 + a_1t_1
\end{equation}
We integrate linear acceleration (\ref{accel}), resulting in a parabolic 
velocity curve. The velocity curve describes the tempo (in beats per minute) 
with respect to time.
\begin{equation}
	\label{bpm}
    \dot{x}_1 = \dot{x}_0 + a_0t_1 + \frac{a_1t_1^2}{2}
\end{equation}
Integrating velocity (\ref{bpm}) gives us a function describing the number 
of beats elapsed with respect to time.
\begin{equation}
	\label{beats-elapsed}
	x_1 = x_0 + \dot{x}_0t_1 + \frac{a_0t_1^2}{2} + \frac{a_1t_1^3}{6}
\end{equation}
With equations (\ref{bpm}) and (\ref{beats-elapsed}), we can solve for our 
two unknowns, $a_0$ and $a_1$. First we solve both equations for $a_1$:
\begin{displaymath}
    \label{a1-solution}
    a_1=
    \frac{-2}{t_1^2}(\dot{x}_0-\dot{x}_1 + a_0t_1)=
    \frac{-6}{t_1^3}(\dot{x}_0-x_1 + \frac{a_0t_1^2}{2})
\end{displaymath}
Assuming $t_1 \neq 0$, we solve this system of equations for $a_0$:
\begin{equation}
	\label{a0-result}
	a_0=\frac{6x_1-2t_1(\dot{x}_1+2\dot{x}_0)}{t_1^2}
\end{equation}
Evaluating (\ref{a0-result}) with our constants gives us our starting
acceleration. Once we have $a_0$ we can solve (\ref{bpm}) for $a_1$, and 
evaluate (\ref{bpm}) with $a_1$ and $a_0$ to describe our changing tempo 
with respect to time.

\section{Implementation}
\label{sec:polytempic-implementation}

Bryn Bliska Tide

\section{Polytempic Music and Polymetric Music}
\label{sec:polytempic-vs-polymetric}
It is easy to confuse poly tempic music with polymetric music.  Pieces
where multiple meter
Other notable composers
of the time include John Cage, who wrote wrote what he called, Chance
Music, and Stockhausen, Boulez and Berio who wrote aleatoric music.



Eliot Carter: Polymetric Modulation. 
Steve Reich: Piano Phase
Paper: realtime representation of 
Paper: Stochos: Software for Real-Time Synthesis of Stochastic Music

John Cage: Chance Music
Karlheinz Stockhausen, Pierre Boulez, Luciano Berio: Aleatoric music

\section{Contribution}
\label{sec:polytempic-contribution}

and finishes the tempo transition slightly different tempos.

To audition stochastic tempo transitions, we can create software tools
that promt musicians with visual cues or aural cues, 
More flexible Digital Audio Workstations (DAWs) like Reaper and
Digital Performer include workarounds for auditioning simultaneous
tempo. 

For example we 

Xenakis was among a group of
20th century composers who were searching for ways to push the
boundaries of established music composition.


%%% Local Variables:
%%% mode: latex
%%% TeX-master: "CharlesHolbrow_MAS_Thesis"
%%% End:
