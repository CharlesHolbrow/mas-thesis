\documentclass{tufte-book}

\hypersetup{colorlinks}% uncomment this line if you prefer colored hyperlinks (e.g., for onscreen viewing)

%%
% Book metadata
\title{Hypercompression}
\author{Charles Holbrow}
%\publisher{Publisher of This Book}

%%
% If they're installed, use Bergamo and Chantilly from www.fontsite.com.
% They're clones of Bembo and Gill Sans, respectively.
%\IfFileExists{bergamo.sty}{\usepackage[osf]{bergamo}}{}% Bembo
%\IfFileExists{chantill.sty}{\usepackage{chantill}}{}% Gill Sans

%\usepackage{microtype}

%%
% Symbol for Euro Currency
\usepackage[official]{eurosym}

%%
% For nicely typeset tabular material
\usepackage{booktabs}

%%
% For graphics / images
\usepackage{graphicx}
\setkeys{Gin}{width=\linewidth,totalheight=\textheight,keepaspectratio}
\graphicspath{{graphics/}}

% The fancyvrb package lets us customize the formatting of verbatim
% environments.  We use a slightly smaller font.
\usepackage{fancyvrb}
\fvset{fontsize=\normalsize}

%%
% Prints a trailing space in a smart way.
\usepackage{xspace}

% Inserts a blank page
\newcommand{\blankpage}{\newpage\hbox{}\thispagestyle{empty}\newpage}

\usepackage{units}
% Control the formatting of lists
\usepackage{enumitem}

% Typesets the font size, leading, and measure in the form of 10/12x26 pc.
\newcommand{\measure}[3]{#1/#2$\times$\unit[#3]{pc}}

% Macros for typesetting the documentation
\newcommand{\hlred}[1]{\textcolor{Maroon}{#1}}% prints in red
\newcommand{\hangleft}[1]{\makebox[0pt][r]{#1}}
\newcommand{\hairsp}{\hspace{1pt}}% hair space
\newcommand{\hquad}{\hskip0.5em\relax}% half quad space
\newcommand{\TODO}[1]{\textcolor{red}{\bf TODO:#1}\xspace}
\newcommand{\ie}{\textit{i.\hairsp{}e.}\xspace}
\newcommand{\eg}{\textit{e.\hairsp{}g.}\xspace}
\newcommand{\na}{\quad--}% used in tables for N/A cells

% Number parts and chapters
\setcounter{secnumdepth}{0}

% Title of my thesis project
\newcommand{\thesis}{Hypercompression\xspace}

\begin{document}

% Front matter
\frontmatter

% Full title page
\maketitle

% Contents
\tableofcontents

% Start the main matter (normal chapters)
\mainmatter

\chapter*{Abstract}
\label{ch:abstract}

\marginnote{\TODO{ Abstract is copy/pasted from another section, and
    should probably be re-written from scratch}} While compression of
mono and stereo audio is well documented and
understood,\cite{Giannoulis2012, Blesser1969} surround sound
compression is relatively less explored.  \thesis expands on
traditional audio compression model by adding spatial control. This
design introduces two additional high level spatial parameters:
\textbf{link angle} and \textbf{spread}. These parameters extend the
domain of the traditional compressor to include surround sound spatial
manipulation in addition to dynamics processing, and unlock new
creative possibilities for surround sound designers.


% Introduction
\cleardoublepage
\chapter{Introduction}
\label{ch:introduction}

\begin{fullwidth}
  \newthought{In the 6th century B.C.} Pythagoras is though to have discovered that
  by dividing resonating string into simple mathematical ratios
  produced harmonious intervals, while arbitrary ratios produced
  dissonance. His observation is probably the first of the many
  explicit parallels that have been identified after his time. Today,
  we describe musical pitches, as integers within a given tuning
  system. We describe the tuning system with a mathematical formula
  that relates frequency to pitch. Musical time, rhythm and meter are
  commonly described numerically. Musical Transposition and inversion
  mirror mathematical functions, and borrow their names directly from
  mathematics.
\end{fullwidth}

As computers, amplifiers and electronics become our primary tools for
creating manipulating, and performing music, mathematics and music
have become more interconnected. Nearly every modern musical
recording, broadcast, and stream is the sumation of many digital
recordings that have been individually discretized, sampled
mathematically encoded, and digitally processed numerous times before
ever reaching our ears. We might be tempted to describe music as
applied mathematics, but doing so betrays a fundamental quality of
music: Musical beauty does not correspond to mathematical elegance or
accuracy. A musicain will diverge from a musical score to accomplish a
particular artistic objective. A vocalist does not abruptly change a
pitch, but gently and carefuly lands on a pitch. A jazz musician might
play slightly \emph{behind the beat}. A classical performer knows how
to hold a fermata \emph{just long enough}. These intentional human
artifacts are characterised more by a \emph{feeling} than by a
formula.

The computer's inability to understand feeling, opened possibilities
for new genres of music like EDM\sidenote[][-30mm]{EDM (Electronic
  Dance Music) features formulaic grooves perfectly locked to a grid,
  and often aggressive use of digital pitch correction exagerating a
  robotic quality.}, Black MIDI\sidenote[][-12mm]{Black MIDI is a musical genre
  that uses low fidelity audio samplers with a large number of midi
  notes over a short time. A single 3 minute Black MIDI track is
  likely to have over 100,000 MIDI notes. The name refers to the solid
  black appearance of the piano score.}, and Demoscene
Music\sidenote{Demoscene music celebrates the computationally
  efficient digital synthesis of electronic music.}. These styles of
music feature (rather than fix) the unfeeling nature of computers. If
we want computer to produce truely expressive music we must try to
capture percieved feelings formulaically, and instruct compters to
reproduce them.


\section{Hypercompression}
\label{sec:hypercompression-intro}
We usually think of compression in terms of \emph{reduction}: We use
data compression to reduce bit-rates and file sizes, audio compression
to reduce dynamic range. Record labels even use compression as a
weapon in the \emph{loudness war}\cite{Deruty2014a}, resulting in some
of today's music recordings utilizing no more dynamic range than a
1909 Edison cylinder.\cite{Katz2007} A deaper study of compression
reveals more subtle and artistic applications. A skilled audio
engineer applies compression to audio with the intention to improve
intelligibility, augment articulation, smooth a performance, shape
transients, extract ambience, de-ess vocals, balance multiple signals,
or even add distortion.\cite{Case2007} At its best, the compressor is
a tool for \emph{temporal shaping} first, and not a tool for \emph{dynamic
  reduction} second.

\thesis expands the traditional model of a compressor to include
\emph{spatial shaping}. While unconventional, spatial processing is a
very natural fit for the compression paradigm. We can think of sound
as a medium that exists in time just as easily as we can think of
sound as soemthing that exists in space.\sidenote{Converting
  measurement of sound from the cycles per second (in the temporal
  domain) to wavelength (in the spatial domain) is common objective in
  acoustics and audio engineering practices. See \textit{The Sound
    Reinforcement Handbook} by G. Davis for examples.} The
Hypercompressor described in detail in \autoref{ch:hypercompressor}.

\paragraph{Performance}
\TODO{Comments about live performance}

\section{Context}
\label{sec:context}

\newthought{The Phillips Pavilion} stands out as an inspiration, a
reference, and a guide for projects that successfully disregard the
conventional disciplinary approach to the creative design process. The
pavilion was a commissioned by Phillips Corporation for the Brussels
World's Fair in 1958.\cite{Zvonar1999} When the architectural offices of
Le Corbusier received the commission, Le Corbusier replied, saying:
``I will not make a pavilion for you but an Electronic Poem and a
vessel containing the poem; light, color image, rhythm and sound
joined together in an organic synthesis.''\cite{Lopez2011} Indeed,
the pavilion embodied Le Corbusier's description, and the resulting
Gesamtkunstwerk included:\cite{Lombardo2009}
\begin{enumerate}
\item A concrete pavilion, designed by architect and composer Iannis
  Xenakis
\item \textit{Interlude Sonoire} (later renamed \textit{Concret PH}), a
  tape music composition by Iannis Xenakis, approximately 2 minutes
  long, played while the audience transitioned
\item A three channel, 8 minute tape music composition, by French-born
  composer Edgard Var\`{e}se
\item A system for spatialized audio across at least 350 loudspeakers
  distributed throughout the pavilion
\item An assortment of visual effects, designed by Le Corbusier in
  collaboration with Philips art director Louis Kalff
\item Video consisting mostly of black and white still images,
  projected on two walls inside the pavilion
\item A system for synchronizing playback of audio and video playback,
  with light effects and audio spatialization throughout the
  experience
\end{enumerate} 

It is a little surprising that Le Corbusier chose Iannis Xenakis, a
young engineer with no formal architectural training to design the
pavilion building, and compose a part of the music. While the
relationship between Le Corbusier and Xenakis would deteriorate as a
result of their work on Philips Pavilion, there was probably no one
better equipped than Xenakis to merge the fields of music, mathematics
and architecture. To understand Xenakis' impact on the project, we
have to review the influences and experiences that led him to Paris
and to Le Corbusier's architecture firm in 1947.

\subsection{The Influence of Iannis Xenakis}
\label{sec:influence-xenakis}

Xenakis was born in Romania in 1922, and moved to Greece in 1932. He
left school at the age of 16, and spent his time reading about
astronomy, archeology, ancient literature, and
mathematics.\cite[]{Hoffmann2015} He was admitted to the
Polytechnic Institute in Athens in 1940, where he studied music,
counterpoint, and engineering. While attending the Polytechnic
Institute, he fought with the resistance against the Nazi Invasion in
Greece. He was jailed and tortured multiple times for his involvement
with the resistance, and eventually sentenced to death for terrorism,
but managed to escape.\cite[]{Simms2014} In 1946 he received his
degree in engineering from the Polytechnic Institute, and left Greece
the following year, using a forged passport.\cite[]{Prendergast2014}

Xenakis began working for Le Corbusier in 1948, but he continuted to
study and write music. While Xenakis was searching for a music mentor,
he approached Oliver Messiaen\sidenote{Messiaen was a well known
  french composer known for rhythmic complexity, and transcribing
  birdsong into his music.}, and asked if he should study harmony or
counterpoint. Messiaen later described his conversation with
Xenakis:

\begin{quotation}``I think one should study harmony and
  counterpoint. But this was a man so much out of the ordinary that I
  said: No, you are almost 30, you have the good fortune of being
  Greek, of being an architect and having studied special
  mathematics. Take advantage of these things. Do them in your
  music.''\cite{Service2013}
\end{quotation}

Ultimately, Messiaen was rejecting Xenakis as a student, but we can
see how Xenakis did draw from disparate skills in his composition. The
score for his 1945 composition \textit{Metastasis}
(figure~\ref{fig:metastasis}), resembles an architectural blueprint as
much as it does a music score.

\begin{figure*}[h]
  \includegraphics[width=\linewidth]{XenakisMetastasis.jpg}
  \caption{Excerpt from Iannis Xenakis' composition,
    \textit{Metastasis} (1954), measures 309-314. This score was then
    transcribed to sheet music for the orchestral performance.}
  \label{fig:metastasis}
\end{figure*}

In 1956, Le Corbusier was focusing much of his attention on a larger
project; Chandigarh, a city in India on the edge of the Himalayan
plains. When he was approached by Louis Kalff and asked to build a
pavillion for the 1958 World's Fair in Brussels, he immediately
accepted. Kalff wanted the pavilion to showcase the sound and lighting
potential of Philips' technologies. Le Corbusier determined that
the shape of the building should resemble a stomach, with the audience
entering through one entrance, and exiting through another. Thinking
the design of the entire city of Chandigarh would be the masterpiece
of his Architectural career,\cite{Flint2013} he delegated the design
of the pavilion building to Xenakis.\cite{Clarke2012}

\paragraph{Polymath} The architectural evolution of the pavillion from
Le Corbusier's early designs (figure~\ref{fig:le-corbusier-sketch})
through Xenakis' iterations (figure~\ref{fig:xenakis-draw}),
illustrates the impact that Xenakis had on the project. The
\textit{Philips Technical Review}\cite{philips1958} gives a
wonderfully detailed account of Xenakis' process:
\begin{enumerate}
\item Xenakis was aware that parallel walls, or concave spherical
  walls would negatively impact audio perceptibility due to repeated
  or localized acoustic reflections.
\item To acomodate musical purpose of the space he decided to explore
  surfaces with varying curvature...
\item ...leading him to consider ruled
  surfaces\sidenote{\TODO{Explain}} such as the conoid and hyperbolic
  parabaloid.
\end{enumerate}
We see Xenakis utilizing the drafting skills that he learned at the
Polytechnic Instutute and continuted to develop while working with Le
Corbusier. He also understood the mathematical focumation of the ruled
surfaces that make up the structure. These surfaces even look familiar
from the Metastasis score (figure~\ref{fig:metastasis}).

\TODO{The important thing is to understand how Xenakis could not have
synthesized such a progressive and creative structure if he had been
trained in only a single discipline... we see the value of his
interdisciplinary background}

\begin{figure*}[h]
  \includegraphics[width=\linewidth]{LeCorbusierDraw.pdf}
  \caption{Le Corbusier's design sketches for the Philips Pavilion,
    September \textendash{} October, 1956 (\textcircled{c} 2012
    Artists Rights Society, New York/ADAGP, Paris/FLC)}
  \label{fig:le-corbusier-sketch}
\end{figure*}

\begin{figure*}[h]
  \includegraphics[width=\linewidth]{PhilipsDrawings.jpg}
  \caption{Xenakis' early drawings of the Philips Pavilion as
    documented in the \textit{1958 Philips Technical Review} \TODO{cite}}
  \label{fig:xenakis-draw}
\end{figure*}


\section{Architecture and Music in Space and Time}
\label{sec:introduction-conclusion}

In his 1963 book \textit{Formalized Music}, Xenakis describes how
developments in music theory mimic equivalent developments in
philosophy, mathematics, and the sciences. Polato, for example,
believed that all events transpire as determinted by cause and
effect. While Plato and Aristotle both described causality in their
writing, it was not until the 17th century that controlled experiments
and mathematics coroborated the theory.\sidenote{In 1687, Isaac Newton
  published \textit{Philosophi\ae{} Naturalis Principia Mathematica}
  (\textit{Mathematical Principles of Natural Philosophy}), in which
  he compiled the 3 laws of motion that set the foundation for the
  study of \emph{classical mechanics}.}  Similarly, historical music
follows deterministic progressions, and music theory employs causal
rules to describe counterpoint, tonality, and harmonic movement.

Causality was largely used to describe physical phenomena until the
19th century when statistical theories in physics began to include
probabilistic notions.\sidenote{The Maxwell-Boltzmann distribution,
  which was first derived by James Clerk Maxwell in 1860, describes
  the probability distribution for the speed of a particle within an
  idealized gas. For more see
  \url{http://plato.stanford.edu/entries/statphys-statmech/}} Xenkis
noticed that more contemporary fields like \emph{probability theory}
and \emph{fuzzy logic} generalize and expand on the antecedent
theories of causality.

Xenakis thought that music composition should naturally follow the
progression that physics did, with the theory of music generalizing
and expanding on causal rules that had existed previously. Indeed,
starting in the late 19th century, and early 20th century, composers
like Strauss and Debussy began to bend the existing rules of music
theory, composing music that branched away from the causal and tonal
theories of the time. With the rise of serialism\sidenote{\TODO{Brief
    Serialism Explianation}} and indeterminate
music\sidenote{\TODO{Brief Indeterminate Music explaination}},
composers such as Strauss, Debussy, Stockhausen, Boulez, John Cage,
Aaron Copland, and B\'{e}la Bart\'{o}k began to use probability and
chance in composition, the same way that physicists were using
probability to describe the material world. However, to Xenakis'
mathematical mind, serial music was no less causal than the music it
intended to supersede. He described serial music as embodying
``virtually absolute determinism.''\cite{xenakis1992formalized}
Xenakis saw music theory as a sub-set of mathematics and
algebra. While musicians have a different vocabulary, they also use
mathematical principles to describe and compose music. Because he
understood mathematics as well as music, he was able to identify how
even in serialism and indeterminate music, composers were only utilizing a
small subset of algebraic theory. In his own music, Xenakis wanted to
generalize the and expand the causal framework that musicians and
theorists had been using to compose and understand music. This
paralleled the developments in phyisics and mathematics that helped
him to form his opinions about music theory.  As a nod to
\emph{chance} or \emph{stochos} xenakis coined the term
\emph{stochastic music} to describe this development.

In the Spring of 1976, Xenakis was defending his doctoral thesis at
the University of Paris. A translation of his defense includes this
statement:
\begin{quotation}
  ``The artist-conceptor will have to be knowledgeable and inventive
  in such varied domains as mathematics, logic, physics, chemistry,
  biology, genetics, palentology (for the evolution of forms), the
  human sciences, and history; in short, a sort of
  \emph{universality}, but one based opon, guided by and oriented
  toward forms and architectures.'' \cite{russolo1986art}
\end{quotation}

From Xenakis' drawings we can deduce that he used the same tools,
skills, and philosophy to imagine and concive both music and
space. His approach elevated both forms by blurring the distinction
between the two. Maybe if we had kept using pen and paper to design
buildings and write music, the reality today would be closer to the
ideal that Xenakis imagined. Today, software for creating architecture
and composing music both favor corners to curves, and static pitches
to glissandi. More importantly, the software skills that we use to
design and maniuplate space are not transferable to the composition of
music.

This is where I want to make a contrubution. By drawing from music,
mathematics, computer science, acoustics, audio engineering and
mixing, sound reinforcement, multimedia production, and live
performance, we can create tools that allow us to indiscriminately
compose with space and sound.

\section{Universality}
\label{sec:universality}

At the MIT Media Lab, we celebrate the study and practice of projects
that exist outside of established academic disciplines. The Media Lab
(and the media) have described this approach as interdisciplinary,
cross-disciplinary, anti-disciplinary, or post-disciplinary -
emphasizing the clich\'{e} that traditional academics must become
experts in their field, and while narrowing their focus, they learn
\textit{more and more about less and less}, and eventually know
\textit{everything about nothing}.  \thesis is truly a Media Lab
project. It documents the creative process throughout the design,
development, and performance of a new type of audio signal
processor. In doing so, it draws from music, mathematics, computer
science, acoustics, audio engineering and mixing, sound reinforcement,
multimedia production, and live performance. How can we describe and
document a project with such broad subject material?  

Within a single
discipline, there is an accepted hierarcy of concepts, and we are
expected to develop a \emph{deep} understanding that penetrates this
hierarchy. We expect students to be literate in algebra, geometry and
calculus before studying physics. When we describe a physics problem,
we depend on an established collection of language, notation, and
theory.

This example reveals the curious tension between breadth and
depth: The \textit{depth} of a disciplinary approach provides the
language and abstraction that enable us to describe content and
communicate at a high level. Depth is essential for solving
non-trivial problems. However, solutions to complex real-world
questions always span multiple disciplines.


It appears we need breadth \emph{and} depth simultaneously. 

The challenge is to describe \thesis so that it is accessible to
readers from all disciplines. This thesis proposes the motivation for
documenting media that exists outside of established disciplines. It
proposes a strategy for such documentation, and employs this strategy
by documenting the theory, implementation, and application of \thesis.


\TODO{This could segway into ``strategy'' section. I've already
  written some of this. It could also segway into ``motivation'', }

\chapter{The Hypercompressor}
\label{ch:hypercompressor}

This is how it works

%%% Local Variables:
%%% mode: latex
%%% TeX-master: "CharlesHolbrow_MAS_Thesis"
%%% End:


\backmatter

\clearpage
\chapter*{Acknowledgements}
\label{ch:acknowledgements}
\begin{fullwidth}
  \noindent Thanks to Tod Machover for welcoming me into the Opera of
  the Future, for unending support, encouragement, and mentorship; for
  sharing your process, and for support integrating \thesis into
  \textit{Of Experience}.

\vspace{5mm}
\noindent Thanks to my readers James Andy Moorer and Joe Paradiso for
kind, and expert input, support, and guidance.

\vspace{5mm}
\noindent Thanks to Professor Alex Case for inspiring my love
for music, audio engineering, and illuminating the magical subtleties
of dynamic range compression, and for (only slightly begrudgingly)
writing me 1000 recommendations letters. No one is better suited to be
President of the Audio Engineering Society.

\vspace{5mm}
\noindent Thanks to Wonshik Choi and Niyom Lue for your infinite
patience, guidance, and for welcoming me to MIT in 2008. Only you
could have taught a music major to enjoy linux, DSP, and spectroscopy.

\vspace{5mm}
\noindent Thanks to Markus Phillips and Shawn Drost for directly and indirectly
giving me confidence as a software developer.

\vspace{5mm}
\noindent Thanks to my UMass Lowell Piano teachers for taking chance
with me, and putting up with me for four years. Anthony Mele,
Elizabeth Skavish, Bonnie Anderson, and Thomas Stumpf - You believed
in me before I did. I'm probably the only student ever who was lucky
enough to study with all four of you.

\vspace{5mm}
\noindent  Thanks to Gene Atwood for being considerate of everyone, and showing
me how important that is\ldots{} And for screaming in to a microphone when
I needed some screams.

\vspace{5mm}
\noindent Thanks to my roommates at The Red House, and Brainerd
Contextual Cooperative for love and support and home. For feeding me
lots of meals, for perspective, and for tolerating my student
lifestyle.

\vspace{5mm}
\noindent Thanks to my team at the MIT Media Lab. Ben Bloomberg for
being my peer and my mentor at the same time. Thanks for inviting me
into the Opera group in 2008 and again in 2014, teaching me a
tremendous amount about audio, and using your live sound prowess to
make \textit{De L'Exp\'{e}rience} sound amazing live. Bryn Bliska, for
conceiving Tempo Toy, which I only slightly stole and turned into
stochastic polytempic modulation, for letting me work out the maths,
and for humility, warmth and musical brilliance. Thanks to Rebecca
Kleinberger for infinite support and insight into everything from math
to hedgehogs to humans. David for peace, intensity, and robots. Akito
for your inspiring attitude, wisdom, projects, and humility. Kelly for
unending support, consideration, and sense of humor. Simone
for taking the ``im'' out of impossible (over and over and over
again).

\vspace{5mm}
\noindent Thanks to Helen Corless for being amazing supportive even
when I am in the absolute pits of grad student existence. Thank you
for always reminding me what music is really about, and for
challenging me like no one else can. Your kindness and wisdom make
the world a better place every day.

\vspace{5mm}
\noindent Thanks to my grandparents for leading by example, and teaching
kindness and dedication, and for endless support in education.

\vspace{5mm}
\noindent Thanks to my wonderful siblings: Hilary, Giles, and Felicity
for always inspiring me with kindness, honesty, and wisdom.

\vspace{5mm}
\noindent And thanks to my parents, Gwen and Mark for forcing me to get an
education before I was wise enough to know I wanted one. Thank you for
all your love and support and everything forever.
\end{fullwidth}

%%% Local Variables:
%%% mode: latex
%%% TeX-master: "CharlesHolbrow_MAS_Thesis"
%%% End:


\bibliography{library}
\bibliographystyle{plainnat}

\end{document}


%%% Local Variables:
%%% mode: latex
%%% TeX-master: t
%%% End:
