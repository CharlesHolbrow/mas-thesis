\documentclass{tufte-book}

\hypersetup{colorlinks}% uncomment this line if you prefer colored hyperlinks (e.g., for onscreen viewing)

%%
% Book metadata
\title{Hypercompression}
\author{Charles Holbrow}
%\publisher{Publisher of This Book}

%%
% If they're installed, use Bergamo and Chantilly from www.fontsite.com.
% They're clones of Bembo and Gill Sans, respectively.
%\IfFileExists{bergamo.sty}{\usepackage[osf]{bergamo}}{}% Bembo
%\IfFileExists{chantill.sty}{\usepackage{chantill}}{}% Gill Sans

%\usepackage{microtype}

%%
% Symbol for Euro Currency
\usepackage[official]{eurosym}

%%
% For nicely typeset tabular material
\usepackage{booktabs}

%%
% For graphics / images
\usepackage{graphicx}
\setkeys{Gin}{width=\linewidth,totalheight=\textheight,keepaspectratio}
\graphicspath{{graphics/}}

% The fancyvrb package lets us customize the formatting of verbatim
% environments.  We use a slightly smaller font.
\usepackage{fancyvrb}
\fvset{fontsize=\normalsize}

%%
% Prints a trailing space in a smart way.
\usepackage{xspace}

% Inserts a blank page
\newcommand{\blankpage}{\newpage\hbox{}\thispagestyle{empty}\newpage}

\usepackage{units}
% Control the formatting of lists
\usepackage{enumitem}

% Typesets the font size, leading, and measure in the form of 10/12x26 pc.
\newcommand{\measure}[3]{#1/#2$\times$\unit[#3]{pc}}

% Macros for typesetting the documentation
\newcommand{\hlred}[1]{\textcolor{Maroon}{#1}}% prints in red
\newcommand{\hangleft}[1]{\makebox[0pt][r]{#1}}
\newcommand{\hairsp}{\hspace{1pt}}% hair space
\newcommand{\hquad}{\hskip0.5em\relax}% half quad space
\newcommand{\TODO}[1]{\textcolor{red}{\bf TODO:#1}\xspace}
\newcommand{\ie}{\textit{i.\hairsp{}e.}\xspace}
\newcommand{\eg}{\textit{e.\hairsp{}g.}\xspace}
\newcommand{\na}{\quad--}% used in tables for N/A cells

% Title of my thesis project

\newcommand{\thesis}{Hypercompression\xspace}

\begin{document}

% Front matter
\frontmatter

% Full title page
\maketitle

% Contents
\tableofcontents

% Start the main matter (normal chapters)
\mainmatter

% Introduction
\cleardoublepage
\chapter{Introduction}
\label{ch:introduction}

\begin{fullwidth}
  \newthought{At the Media Lab}, we celebrate the study and practice
  of projects that exist outside of established academic
  disciplines. The Media Lab, and the media have described this
  approach as interdisciplinary, cross-disciplinary,
  anti-disciplinary, or post-disciplinary - emphasizing the clich\'{e}
  that traditional academics must become experts in their field, and
  while narrowing their focus, they learn \textit{more and more about
    less and less}, and eventually know \textit{everything about
    nothing}.  \thesis is truly a Media Lab project. It documents the
  creative process throughout the design, development, and performance of
  a new type of audio signal processor. In doing so, it draws from
  music, mathematics, computer science, acoustics, audio engineering
  and mixing, sound reinforcement, multimedia production, and live
  performance.

\end{fullwidth}

\paragraph{Breadth and Depth} How can we describe a project with such
broad subject material?  A natural tension exists between breadth and
depth: Where the \textit{depth} of a strictly disciplinary approach
provides language and abstractions that enable us to describe content
and communicate at a high level, the \textit{breadth} of the
anti-disciplinary approach invites us to partner with collaborators
with diverse backgrounds and abilities. The challenge is to describe
\thesis so that it is accessible to readers from all disciplines. This
thesis proposes the motivation for documenting media that exists
outside of established disciplines. It proposes a strategy for such
documentation, and employs this strategy by documenting the theory,
implementation, and application of \thesis.

\paragraph{Compression and Hypercompression} A traditional
\marginnote{Be aware of the diference between audio data compression
  and audio dynamic range compression. Data compression is the
  practice of reducing the filesize of digitally encoded audio data. A
  dynamic range compressor is one tool used by audio engineers to
  parametrically manipulate the amplitude of an audio signal. In this
  thesis, \textit{compression} refers to \textit{dynamic range
    compression}.}
dynamic range compressor is one of the most powerful and flexible
tools in the audio engineer's toolkit.  A basic compressor can be
thought of as an automatic volume control that simply reduces the
level of an audio signal when the signal exceeds a threshold. By
carefully adjusting the threshold, the input signals, the reduction
amount, and the rate at which the compressor reacts, a skilled audio
engineer can use compression to increase perceived loudness, improve
intelligibility, augment articulation, smooth a performance, shape
transients, extract ambience, de-ess vocals, balance multiple signals,
and add distortion.\cite{Case2007}

While compression of mono and stereo audio is well documented and
understood,\cite{Giannoulis2012, Blesser1969, Katz2007, Case2007}
surround sound compression is relatively less explored.  \thesis
expands on traditional audio compression model by adding 
spatial control. This design introduces two additional high level
spatial parameters: \textbf{link angle} and \textbf{spread}. These
parameters extend the domain of the traditional compressor to include
surround sound spatial manipulation in addition to dynamics
processing, and unlock new creative possibilities for surround sound
designers.

\paragraph{Performance}
\TODO{Comments about live performance}


\section{Structure}
\label{sec:structure}
\TODO{I'll fill this out more as I progress, but this is basically the
  chapter structure I'm imagining. Building the software likely won't
  have enough content for a whole chapter, so I may merge it with the
  math chapter}
\begin{enumerate}
\item Introduction and  Historical Context % context primes for motivation
\item Motivation
\item Framework for multimedia documentation Approach/Computer Systems
\item Describe the Mathematics. % Note that Wikipedia states that
                                % Xenakis pioneered the use of
                                % Set-theory in music. this is a nice
                                % tie in to Set Builder
                                % Notation, which when seen is ungoogleable
\item Build the Software
\item Live multimedia performance
\item Concert playback recording. 
\item Conclusion/Educational Implications
\end{enumerate}

\section{Context}
\label{sec:context}

\newthought{The Phillips Pavilion} stands out as an inspiration, a
reference, and a guide for projects that successfully disregard the
conventional disciplinary approach to the creative design process. The
pavilion was a commissioned by Phillips Corporation for the Brussels
World's Fair in 1958.\cite{Zvonar1999a} When the architectural offices of
Le Corbusier received the commission, Le Corbusier replied, saying:
``I will not make a pavilion for you but an Electronic Poem and a
vessel containing the poem; light, color image, rhythm and sound
joined together in an organic synthesis.''\cite{Lopez2011} Indeed,
the pavilion embodied Le Corbusier's description, and the resulting
Gesamtkunstwerk included:\cite{Lombardo2009}
\begin{enumerate}
\item A concrete pavilion, designed by architect and composer Iannis
  Xenakis
\item \textit{Interlude Sonoire} (later renamed \textit{Concret PH}), a
  tape music composition by Iannis Xenakis, approximately 2 minutes
  long, played while the audience transitioned
\item A three channel, 8 minute tape music composition, by French-born
  composer Edgard Var\`{e}se
\item A system for spatialized audio across at least 350 loudspeakers
  distributed throughout the pavilion
\item An assortment of visual effects, designed by Le Corbusier in
  collaboration with Philips art director Louis Kalff
\item Video consisting mostly of black and white still images,
  projected on two walls inside the pavilion
\item A system for synchronizing playback of audio and video playback,
  with light effects and audio spatialization throughout the
  experience
\end{enumerate} 

It is a little surprising that Le Corbusier chose Iannis Xenakis, a
young engineer with no formal architectural training to design the
pavilion building, and compose a part of the music. While the
relationship between Le Corbusier and Xenakis would deteriorate as a
result of their work on Philips Pavilion, there was probably no one
better equipped than Xenakis to merge the fields of music, mathematics
and architecture. To understand Xenakis' impact on the project, we
have to review the influences and experiences that led him to Paris
and to Le Corbusier's architecture firm in 1947.

\subsection{The Influence of Iannis Xenakis}
\label{sec:influence-xenakis}

Xenakis was born in Romania in 1922, and moved to Greece in 1932. He
left school at the age of 16, and spent his time reading about
astronomy, archeology, ancient literature, and
mathematics.\cite{Hoffmann2015} He was admitted to the Polytechnic
Institute in Athens in 1940, where he studied music, counterpoint, and
engineering. While attending the Polytechnic Institute, he fought with
the resistance against the Nazi Invasion in Greece. He was jailed and
tortured multiple times for his involvement with the resistance, and
eventually sentenced to death for terrorism, but managed to
escape\cite{Simms2014}. In 1946 he received his degree in
engineering from the Polytechnic Institute, and
left Greece the following year, using a forged
passport.\cite[-24]{Prendergast2014}

Xenakis began working for Le Corbusier in 1948, but he continuted to
study and write music. While Xenakis was searching for a music mentor,
he approached Oliver Messiaen\sidenote{Messiaen was a well known
  french composer known for rhythmic complexity, and transcribing
  birdsong into his music.}, and asked if he should study harmony or
counterpoint. Messiaen later described his conversation with
Xenakis:\cite{Service2013}

\begin{quotation}
  I think one should study harmony and counterpoint. But this was a
  man so much out of the ordinary that I said: \begin{quote}No, you
    are almost 30, you have the good fortune of being Greek, of being
    an architect and having studied special mathematics. Take
    advantage of these things. Do them in your music.\end{quote}
\end{quotation}

Ultimately, Messiaen was rejecting Xenakis as a student, but we can
see how Xenakis did draw from disparate skills in his composition. The
score for his 1945 composition \textit{Metastasis}
(figure~\ref{fig:metastasis}), resembles an architectural blueprint as
much as it does a music score.

\begin{figure*}[h]
  \includegraphics[width=\linewidth]{XenakisMetastasis.jpg}
  \caption{Excerpt from Iannis Xenakis' composition,
    \textit{Metastasis} (1954), measures 309-314}
  \label{fig:metastasis}
\end{figure*}

In 1956, Le Corbusier was focusing much of his attention on a larger
project; Chandigarh, a city in India on the edge of the Himalayan
plains. When he was approached by Louis Kalff and asked to build a
pavillion for the 1958 World's Fair in Brussels, he immediately
accepted. Kalff wanted the pavilion to showcase the sound and lighting
potential of Philips' technologies. Le Corbusier determined that
the shape of the building should resemble a stomach, with the audience
passing through one entrance, and leaving through another. Thinking
the design of the entire city of Chandigarh would be the masterpiece
of his Architectural career,\cite{Flint2013} he delegated the design
of the pavilion building to Xenakis.\cite{Clarke2012}

\begin{figure*}[h]
  \includegraphics[width=\linewidth]{LeCorbusierDraw.pdf}
  \caption{Le Corbusier's design sketches for the Philips Pavilion,
    September \textendash{} October, 1956 (\textcircled{c} 2012 Artists Rights
    Society, New York/ADAGP, Paris/FLC)}
  \label{fig:le-corbusier-draw}
\end{figure*}

\begin{figure*}[h]
  \includegraphics[width=\linewidth]{PhilipsDrawings.jpg}
  \caption{Xenakis' early drawings of the Philips Pavilion as
    documented in the \textit{1958 Philips Technical Review} \TODO{cite}}
  \label{fig:xenakis-draw}
\end{figure*}

\newthought{In 2004}, the Culture 2000 Programme created by the 
European union approved a 99510\EUR{} grant to an Italian Multimedia 
firm for a project called Virtual Electronic Poem (VEP)\cite{eu2004}. 
The project proposed to create a virtual reality simulation in which 
users could experience rendered audio and video of the famous
Gesamtkunstwerk created for the Phillips Pavilion at the 1958 World
Expo in Brussels. 

The goal of the project is included in the Culture 2000 Programme
report: The project will reproduce the experience created by the
Philips pavilion at the Brussels Expo Universelle in 1958.

Give Details? Explain all the parts that were involved with the
original pavilion, explain all the resources that were reviewed to
figure out what the thing actually looked like.

This project was so involved, that the principal investigator coined
the term "Archeology of Multimedia" to describe the experience of
recovering\TODO{cite}.

Virtual reality technology has changed enough between 2004 and 2015,
that reviving the VEP project would probably take an additional
multimedia archeology expedition.

The problem of preservation described above informs the structure and
strategy in this thesis. 

Explain the need for audio systems as computer systems? Cite
Granularity. Describe what fields I draw from. Describe how fields can
go deep, but Anti Disciplinary is unbounded. Describe how this could
only happen at the media lab. Detail how I will granular-ize?

\section{Motivation}
\label{sec:motivation}

  - Sal Khan's talk about students who get stuck. 
  - Talk about the expectations for this paper - what should you
    already understand? Basic Audio theory. 
  - specialization makes it hard for different fields to communicate
  with each other and learn from each other.
  - Think about documentation in terms of Systems, granulatiry
  - Noticed how very simple concepts are portrayed 
  - Leave behind bridges between disciplines
  - set builder notation, and un-google-able
  questions. IE. Gesamtkunstwerk is googleable, and does not need a
  citation. 

\section{Architecture and Music in Space and Time}
\label{sec:introduction-conclusion}

In the Spring of 1976, Xenakis was defending his doctoral thesis at
the University of Paris. A translation of his defense includes this
statement:\cite{russolo1986art}
\begin{quotation}
The artist-conceptor will have to be knowledgeable and inventive in
such varied domains as mathematics, logic, physics, chemistry,
biology, genetics, palentology (for the evolution of forms), the human
sciences, and history; in short, a sort of universality, but one based
opon, guided by and oriented toward forms and architectures.
\end{quotation}


Xenakis wanted to free music from conventions of his time, and he
wanted to free architecture from spatial paradigms. He saw how By
seeing\TODO{...}

From Xenakis' drawings we can deduce that he used the same tools,
skills, and philosophy to imagine and concive both music and
space. His approach elevated both forms by blurring the distinction
between the two. Maybe if we had kept using pen and paper to design
buildings and write music, the reality today would be closer to the
ideal that Xenakis imagined. Today, software for creating architecture
and composing music both favor corners to curves, and static pitches
to glissandi. More importantly, the software skills that we use to
design and maniuplate space are not transferable to the composition of
music.

This is where I want to make a contrubution. By drawing from music,
mathematics, computer science, acoustics, audio engineering and
mixing, sound reinforcement, multimedia production, and live
performance, we can create tools that indiscriminately compose with
space and sound.

\backmatter

\clearpage
\chapter*{Acknowledgements}
\label{ch:acknowledgements}
\begin{fullwidth}
  \noindent Thanks to Tod Machover for welcoming me into the Opera of
  the Future, for unending support, encouragement, and mentorship; for
  sharing your process, and for support integrating \thesis into
  \textit{Of Experience}.

\vspace{5mm}
\noindent Thanks to my readers James Andy Moorer and Joe Paradiso for
kind, and expert input, support, and guidance.

\vspace{5mm}
\noindent Thanks to Professor Alex Case for inspiring my love
for music, audio engineering, and illuminating the magical subtleties
of dynamic range compression, and for (only slightly begrudgingly)
writing me 1000 recommendations letters. No one is better suited to be
President of the Audio Engineering Society.

\vspace{5mm}
\noindent Thanks to Wonshik Choi and Niyom Lue for your infinite
patience, guidance, and for welcoming me to MIT in 2008. Only you
could have taught a music major to enjoy linux, DSP, and spectroscopy.

\vspace{5mm}
\noindent Thanks to Markus Phillips and Shawn Drost for directly and indirectly
giving me confidence as a software developer.

\vspace{5mm}
\noindent Thanks to my UMass Lowell Piano teachers for taking chance
with me, and putting up with me for four years. Anthony Mele,
Elizabeth Skavish, Bonnie Anderson, and Thomas Stumpf - You believed
in me before I did. I'm probably the only student ever who was lucky
enough to study with all four of you.

\vspace{5mm}
\noindent  Thanks to Gene Atwood for being considerate of everyone, and showing
me how important that is\ldots{} And for screaming in to a microphone when
I needed some screams.

\vspace{5mm}
\noindent Thanks to my roommates at The Red House, and Brainerd
Contextual Cooperative for love and support and home. For feeding me
lots of meals, for perspective, and for tolerating my student
lifestyle.

\vspace{5mm}
\noindent Thanks to my team at the MIT Media Lab. Ben Bloomberg for
being my peer and my mentor at the same time. Thanks for inviting me
into the Opera group in 2008 and again in 2014, teaching me a
tremendous amount about audio, and using your live sound prowess to
make \textit{De L'Exp\'{e}rience} sound amazing live. Bryn Bliska, for
conceiving Tempo Toy, which I only slightly stole and turned into
stochastic polytempic modulation, for letting me work out the maths,
and for humility, warmth and musical brilliance. Thanks to Rebecca
Kleinberger for infinite support and insight into everything from math
to hedgehogs to humans. David for peace, intensity, and robots. Akito
for your inspiring attitude, wisdom, projects, and humility. Kelly for
unending support, consideration, and sense of humor. Simone
for taking the ``im'' out of impossible (over and over and over
again).

\vspace{5mm}
\noindent Thanks to Helen Corless for being amazing supportive even
when I am in the absolute pits of grad student existence. Thank you
for always reminding me what music is really about, and for
challenging me like no one else can. Your kindness and wisdom make
the world a better place every day.

\vspace{5mm}
\noindent Thanks to my grandparents for leading by example, and teaching
kindness and dedication, and for endless support in education.

\vspace{5mm}
\noindent Thanks to my wonderful siblings: Hilary, Giles, and Felicity
for always inspiring me with kindness, honesty, and wisdom.

\vspace{5mm}
\noindent And thanks to my parents, Gwen and Mark for forcing me to get an
education before I was wise enough to know I wanted one. Thank you for
all your love and support and everything forever.
\end{fullwidth}

%%% Local Variables:
%%% mode: latex
%%% TeX-master: "CharlesHolbrow_MAS_Thesis"
%%% End:


\bibliography{library}
\bibliographystyle{plainnat}

\end{document}


%%% Local Variables:
%%% mode: latex
%%% TeX-master: t
%%% End:
