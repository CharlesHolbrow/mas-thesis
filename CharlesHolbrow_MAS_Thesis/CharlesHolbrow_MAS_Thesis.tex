\documentclass{tufte-book}

\hypersetup{colorlinks}% uncomment this line if you prefer colored hyperlinks (e.g., for onscreen viewing)

%%
% Book metadata
\title{Hypercompression}
\author{Charles Holbrow}
%\publisher{Publisher of This Book}

%%
% If they're installed, use Bergamo and Chantilly from www.fontsite.com.
% They're clones of Bembo and Gill Sans, respectively.
%\IfFileExists{bergamo.sty}{\usepackage[osf]{bergamo}}{}% Bembo
%\IfFileExists{chantill.sty}{\usepackage{chantill}}{}% Gill Sans

%\usepackage{microtype}

%%
% Symbol for Euro Currency
\usepackage[official]{eurosym}

%%
% For nicely typeset tabular material
\usepackage{booktabs}

%%
% For graphics / images
\usepackage{graphicx}
\setkeys{Gin}{width=\linewidth,totalheight=\textheight,keepaspectratio}
\graphicspath{{graphics/}}

% The fancyvrb package lets us customize the formatting of verbatim
% environments.  We use a slightly smaller font.
\usepackage{fancyvrb}
\fvset{fontsize=\normalsize}

%%
% Prints a trailing space in a smart way.
\usepackage{xspace}

% Inserts a blank page
\newcommand{\blankpage}{\newpage\hbox{}\thispagestyle{empty}\newpage}

\usepackage{units}

% Typesets the font size, leading, and measure in the form of 10/12x26 pc.
\newcommand{\measure}[3]{#1/#2$\times$\unit[#3]{pc}}

% Macros for typesetting the documentation
\newcommand{\hlred}[1]{\textcolor{Maroon}{#1}}% prints in red
\newcommand{\hangleft}[1]{\makebox[0pt][r]{#1}}
\newcommand{\hairsp}{\hspace{1pt}}% hair space
\newcommand{\hquad}{\hskip0.5em\relax}% half quad space
\newcommand{\TODO}[1]{\textcolor{red}{\bf TODO:#1}\xspace}
\newcommand{\ie}{\textit{i.\hairsp{}e.}\xspace}
\newcommand{\eg}{\textit{e.\hairsp{}g.}\xspace}
\newcommand{\na}{\quad--}% used in tables for N/A cells

% Title of my thesis project

\newcommand{\thesis}{Hypercompression\xspace}

\begin{document}

% Front matter
\frontmatter

% Full title page
\maketitle

% Contents
\tableofcontents

% Start the main matter (normal chapters)
\mainmatter

% Introduction
\cleardoublepage
\chapter{Introduction}
\label{ch:introduction}

\begin{fullwidth}
  \newthought{At the Media Lab}, we celebrate the study and practice
  of projects that exist outside of established academic
  disciplines. The Media Lab, and the media have described this
  approach as interdisciplinary, cross-disciplinary,
  anti-disciplinary, or post-disciplinary - emphasizing the clich\'{e}
  that traditional academics must become experts in their field,
  and while narrowing their focus, they learn \textit{more and more
    about less and less}, and eventually know \textit{everything about
    nothing}.
\end{fullwidth}

\newthought{\thesis} is truly a Media Lab project. It draws from
music, mathematics, computer science, acoustics, audio engineering and
mixing, sound reinforcement, multimedia production, and live
performance. A natural tension exists between breadth and depth: Where
the \textit{focus} of a strictly disciplinary approach provides
language and abstractions that enable us to describe content and
communicate at a high level, the \textit{breadth} of the anti-disciplinary
approach invites us to partner with collaborators with diverse
backgrounds and abilities. The challenge is to describe \thesis so
that it is accessible to readers from all disciplines. This document
proposes the motivation for documenting media that exists outside of
established disciplines. It proposes a strategy for such
documentation, and employs this strategy by documenting the theory,
implementation, and application of \thesis.


  - Framed with Historical Context
  - Framework for multimedia documentation Approach/Computer Systems
  - Describe the Mathematics 
  - Build the Software
  - Live multimedia performance
  - Concert playback. 
  - Educational Implications


  - Sal Khan's talk about students who get stuck. 
  - specialization makes it hard for different fields to communicate
  with each other and learn from each other. 
  - Think about documentation in terms of Systems, Granulatiry
  - Noticed how very simple concepts are portrayed 
  - Leave bridges to other disciplines 
  

Tension between depth and breadth

\newthought{In 2004,} the Culture 2000 Programme created by the 
European union approved a 99510\EUR{} grant to an Italian Multimedia 
firm for a project called Virtual Electronic Poem (VEP)\cite{eu2004}. 
The project proposed to create a virtual reality simulation in which 
users could experience rendered audio and video of the famous
Gesamtkunstwerk created for the Phillips Pavilion at the 1958 World
Expo in Brussels. 

The goal of the project is included in the Culture 2000 Programme
report: The project will reproduce the experience created by the
Philips pavilion at the Brussels Expo Universelle in 1958.

Give Details? Explain all the parts that were involved with the
original pavilion, explain all the resources that were reviewed to
figure out what the thing actually looked like.

This project was so involved, that the principal investigator coined
the term "Archeology of Multimedia" to describe the experience of
recovering\TODO{cite}.

Virtual reality technology has changed enough between 2004 and 2015,
that reviving the VEP project would probably take an additional
multimedia archeology expedition.

The problem of preservation described above informs the structure and
strategy in this thesis. 

Explain the need for audio systems as computer systems? Cite
Granularity. Describe what fields I draw from. Describe how fields can
go deep, but Anti Disciplinary is unbounded. Describe how this could
only happen at the media lab. Detail how I will granular-ize?

\section{This it the Title}
\label{sec:this-it-title}
Wow this is the new section@!!!!!


\section{Hypercompression}
How it fits into audio engineering, compression, how it fits into this project. 


\chapter{The Design of Tufte's Books}
\label{ch:tufte-design}


\newthought{The pages} of a book are usually divided into three major
sections: the front matter (also called preliminary matter or prelim), the
main matter (the core text of the book), and the back matter (or end
matter).

\section{Typefaces}\label{sec:typefaces1}\index{typefaces}
\index{fonts|see{typefaces}}

Tufte's books primarily use two typefaces: Bembo and Gill Sans.  Bembo is used
for the headings and body text, while Gill Sans is used for the title page and
opening epigraphs in \thesis.

\chapter[Chapter2]{Chapter 2!!}
\label{ch:ch2}

\section{Sidenotes}\label{sec:sidenotes}
One of the most prominent and distinctive features of this style is the
extensive use of sidenotes.

\section{References}
References are placed alongside their citations as sidenotes,
as well.  This can be accomplished using the normalcite command

\section{Figures and Tables}\label{sec:figures-and-tables}
Images and graphics play an integral role in Tufte's work.

\section{Full-width text blocks}

\section{Typography}\label{sec:typography}

\subsection{Typefaces}\label{sec:typefaces}\index{typefaces}
If the Palatino, \textsf{Helvetica}, and \texttt{Bera Mono} typefaces are installed, this style
will use them automatically.  Otherwise, we'll fall back on the Computer Modern
typefaces.

\subsection{Letterspacing}\label{sec:letterspacing}
This document class includes two new commands and some improvements on
existing commands for letterspacing.

\section{Document Class Options}\label{sec:options}

\chapter{Troubleshooting and Support}
\label{ch:troubleshooting}

\section{Errors, Warnings, and Informational Messages}\label{sec:tl-messages}
The following is a list of all of the errors, warnings, and other messages generated by the classes and a brief description of their meanings.

\backmatter

\bibliography{library}
\bibliographystyle{plainnat}

\end{document}


%%% Local Variables:
%%% mode: latex
%%% TeX-master: t
%%% End:
