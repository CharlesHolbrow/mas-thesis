%%%
%% Charles' Hack
%% allow line breaks after hyphens in urls
\PassOptionsToPackage{hyphens}{url}

% openany - Dont foce chapters to start on on right hand page
% oneside - put all page numbers in the upper right, not outer corner
\documentclass[openany, oneside]{tufte-book} 

\hypersetup{colorlinks} % uncomment this line if you prefer colored hyperlinks (e.g., for onscreen viewing)

%%
% Book metadata
\title{Hypercompression}
\author{Charles Holbrow}
%\publisher{Publisher of This Book}

%%
% If they're installed, use Bergamo and Chantilly from www.fontsite.com.
% They're clones of Bembo and Gill Sans, respectively.
%\IfFileExists{bergamo.sty}{\usepackage[osf]{bergamo}}{}% Bembo
%\IfFileExists{chantill.sty}{\usepackage{chantill}}{}% Gill Sans

%\usepackage{microtype}

%%
% Symbol for Euro Currency
\usepackage[official]{eurosym}

%%
% Degree symbol \degree
\usepackage{gensymb}

%%
% For nicely typeset tabular material
\usepackage{booktabs}

%%
% For graphics / images
\usepackage{graphicx}
\setkeys{Gin}{width=\linewidth,totalheight=\textheight,keepaspectratio}
\graphicspath{{graphics/}}

% The fancyvrb package lets us customize the formatting of verbatim
% environments.  We use a slightly smaller font.
\usepackage{fancyvrb}
\fvset{fontsize=\normalsize}

%%
% Prints a trailing space in a smart way.
\usepackage{xspace}

% Inserts a blank page
\newcommand{\blankpage}{\newpage\hbox{}\thispagestyle{empty}\newpage}

\usepackage{units}

% Control the formatting of lists
\usepackage{enumitem}

% tex exported by Mathematica wants us to use these
\usepackage{amsmath, amssymb, graphics, setspace}
\newcommand {\mathsym}[1]{{}}
\newcommand {\unicode}[1]{{}}

% Typesets the font size, leading, and measure in the form of 10/12x26 pc.
\newcommand{\measure}[3]{#1/#2$\times$\unit[#3]{pc}}

% Macros for typesetting the documentation
\newcommand{\hlred}[1]{\textcolor{Maroon}{#1}}% prints in red
\newcommand{\hangleft}[1]{\makebox[0pt][r]{#1}}
\newcommand{\hairsp}{\hspace{1pt}}% hair space
\newcommand{\hquad}{\hskip0.5em\relax}% half quad space
\newcommand{\TODO}[1]{\textcolor{red}{\bf TODO: #1}\xspace}
\newcommand{\ie}{\textit{i.\hairsp{}e.}\xspace}
\newcommand{\eg}{\textit{e.\hairsp{}g.}\xspace}
\newcommand{\na}{\quad--}% used in tables for N/A cells

% How many letters must be in a syllable to allow hyphenation
\lefthyphenmin6   % default 2
\righthyphenmin5 % default 3

 % This block sets up a command for printing an epigraph with 2
 % arguments - the quote and the author
\newcommand{\openFM}[3]{
\begin{fullwidth}
\rmfamily\huge

\noindent\allcaps{\textbf{#1}}\\ % The title
\rmfamily\LARGE
\noindent{#2} % The subtitle
\begin{spacing}{5}
\sffamily
\end{spacing}
\noindent{#3}
\end{fullwidth}
}

% This block sets up a command for printing an epigraph with 2
% arguments - the quote and the author
\newcommand{\openAFM}[3]{
\begin{fullwidth}
\rmfamily\huge

\noindent\allcaps{\textbf{#1}}\\ % The title
\rmfamily\LARGE
\noindent{#2} % The subtitle
\begin{spacing}{2.5}
\sffamily
\end{spacing}
\noindent{#3}
\end{fullwidth}
}

% Number parts and chapters
\setcounter{secnumdepth}{1}

% Fix really stupid Tufte LaTeX thing
\morefloats


% Title of my thesis project
\newcommand{\thesis}{Hypercompression\xspace}
\newcommand{\subtitle}{Stochastic Musical Processing}
\newcommand{\polytempic}{Stochastic Tempo Modulation\xspace}
\newcommand{\refmod}{Reflection Visualizer\xspace}

\begin{document}

% Front matter
\frontmatter
% Full title page
\openFM{\thesis}{\subtitle}{Charles J. Holbrow}

\noindent Bachelor of Music\\
\noindent University of Massachusetts Lowell, 2008\\

\vspace{30mm}

\begin{raggedright}
\noindent Submitted to the Program~in~Media~Arts~and~Sciences,\\
School~of~Architecture~and~Planning, in partial fulfillment\\
of the requirements for the degree of\\
\textbf{Master~of~Science in~Media~Arts~and~Sciences}\\
at the \textbf{Massachusetts~Institute~of~Technology} \\
\noindent September 2015\\
\noindent \textcircled{c}~2015~Massachusetts~Institute~of~Technology. All rights reserved.\\
\end{raggedright}

\begin{fullwidth}
\mbox{ }\\
\mbox{ }\\
\mbox{ }\\


\noindent Author: \hfill CHARLES HOLBROW\vspace{3pt}\hrule\vspace{6pt}
\flushright Program in Media Arts and Sciences\\
\flushright August 7, 2015\\
\mbox{ }\\
\mbox{ }\\
\mbox{ }\\
 
\noindent Certified by: \hfill TOD MACHOVER\vspace{3pt}\hrule\vspace{6pt}
Muriel R. Cooper Professor of Music and Media\\
Program in Media Arts and Sciences\\
Thesis Supervisor\\
\mbox{ }\\ 
\mbox{ }\\

\noindent Accepted by: \hfill PATTIE MAES\vspace{3pt}\hrule\vspace{6pt}
Academic Head\\
Program in Media Arts and Sciences\\

\thispagestyle{empty}  % Empty heads and feet - no page numbers.
\end{fullwidth}
\clearpage
%%% Local Variables:
%%% mode: latex
%%% TeX-master: "CharlesHolbrow_MAS_Thesis"
%%% End:


\clearpage
\chapter*{Abstract}
\label{ch:abstract}
The theory of stochastic music proposes that we of music as a vertical
integration of mathematics, the physics of sound, psychiacoustics, and
traditional music theory. In \textit{\thesis, \subtitle} we explore
the design and implementation of three innovative musical projects
that build on a deep vertial integration of science and technology in
different ways: \refmod, \polytempic, and The Hypercompressor. 
The \refmod introduces an interface for quickly sketching abstract
architectural and musical ideas. \polytempic proposes a mathematical
approach for composing previously inaccesable polytempic music. The
Hypercompressor describes new technique for manipulating music in
space and time. For each project, we examine how stochastic theory can help us
discover and explore new musical possibilites, and we discuss
advantages and shortcommings of this approach.

\vspace{15mm}
\begin{tabbing}
Thesis Supervisor: \=   TOD MACHOVER \\
 \> Muriel R. Cooper Professor of Music and Media \\ 
 \> Program in Media Arts and Sciences \\ 
\end{tabbing}

%*******************************************************
% Readers
%*******************************************************

\clearpage
\openAFM{\thesis}{\subtitle}{Charles Holbrow}
\begin{fullwidth}
\mbox{ }\\
\mbox{ }\\
\mbox{ }\\ 
\vfill
\noindent The following person served as a reader for this thesis:\\
\vspace{10mm}

\noindent Thesis Reader: \hfill Joseph A. Paradiso\vspace{3pt}\hrule\vspace{6pt}
\flushright Associate Professor of Media Arts and Sciences\\
\flushright Program in Media Arts and Sciences\\
\flushright Massachusetts Institute of Technology\\

\thispagestyle{empty}  % Empty heads and feet - no page numbers.
\end{fullwidth}
 
\clearpage 
\openAFM{\thesis}{\subtitle}{Charles Holbrow}
\begin{fullwidth}
\mbox{ }\\
\mbox{ }\\
\mbox{ }\\ 
\vfill
\noindent The following person served as a reader for this thesis:\\
\vspace{10mm}

\noindent Thesis Reader: \hfill James A. Moorer\vspace{3pt}\hrule\vspace{6pt}
\flushright Principal Scientist\\
\flushright Adobe Systems, Incorporated\\
\thispagestyle{empty}  % Empty heads and feet - no page numbers.
\end{fullwidth}
 

%%% Local Variables:
%%% mode: latex
%%% TeX-master: "CharlesHolbrow_MAS_Thesis"
%%% End:




\mainmatter
\tableofcontents
\cleardoublepage
\chapter{Introduction}
\label{ch:introduction}
In his 1963 book, \textit{Formalized Music}, Composer, Engineer, and
Architect, Iannis Xenakis described the foundation for his own
reinterpretation of conventional music theory:
\begin{quotation}
  ``All sound is an integration of grains, of elementary sonic
  particles, of sonic quanta. Each of these elementary grains has a
  threefold nature: duration, frequency, and intensity.''
\end{quotation}
Instead of using high-level musical concepts like pitch and meter to
compose music, Xenakis posited, only the three elementary qualities
(frequency, duration and intensity) are necessary. Because it is
impractical to describe sounds as the sum of hundreds or thousands of
elementary sonic particles, he proposed that statistical and probabilistic
models should be used to define sounds at the macroscopic
level. Additionally, similar mathematical models should describe other
compositional concepts including rhythm, form, and melody. Xenakis
named this new style \textit{stochastic}, and he considered it a
generalization of existing music theory: High-level musical constructs
such as melody, harmony, and meter are mathematical abstractions of
the elementary sonic particles, and alternative abstractions such as
non standard tuning should exist as equals within the same mathematical
framework. Music composition, he claimed, requires a \emph{deep}
understanding of the mathematical relationship between sonic elements
and musical abstractions, and music composition necessarily involves
the formulation of new high-level musical constructs built from the
low-level sound elements.

\paragraph{}Xenakis' choice of frequency,
duration, and intensity as the three sonic elements is rational. Any
audio waveform can be represented as the sum of a (possibly infinite)
number of periodic sine waves. However, a waveform is not the same as
music, or even the same as sound. Sound is three-dimensional; sound
has a direction; sound exists in space.
% Given a propagation medium, and a time duration, sound travels a 
% certain distance in space. Given any distance in space, the
% wavelength
% of some frequency fills that space. 
The three projects described in this thesis each use stochastic
music as a base: \refmod, \polytempic, and \thesis. Each project is
directly inspired and influenced by Xenakis, but departs from his
foundational beliefs by treating time and space as equals,
and as true elements of music.


\section{\refmod}
\label{sec:refmod-intro}
\newthought {Music and space} are intimately connected. The first
project, described in \autoref{ch:ref-mod}, explores how we can
compose music using acoustic reflections in architectural space as a
medium. \refmod is a software tool that lets us design and experiment
with abstract acoustic lenses or ``sound mirrors'' in two
dimensions. It is directly inspired by the music and architecture of
Iannis Xenakis, a 20th century composer, music theorist, architect,
and engineer. His collected works provide guidance and perspective to
all the projects in this thesis.

\section{\polytempic}
\label{sec:polytempic-intro}

\newthought{Music and Time} are inseparable. All music flows through
time and depends on temporal constructs - the most common being meter
and tempo. Accelerating or decelerating tempi are common in many
styles of music, as are polyrhythms.  Music with multiple simultaneous
tempi or \textit{polytempic music} is less common, but still many
examples can be found. Fewer examples of music with simultaneous tempi
that shift relative to each other exist, however, and it is difficult
for musicians to accurately perform changing tempi in
parallel. Software is an obvious choice for composing complex and
challenging rhythms such as these, but existing compositional software
makes this difficult. \polytempic offers a solution to this challenge
by describing a strategy for composing music with multiple
simultaneous tempi that accelerate and decelerate relative to each
other. In \autoref{ch:polytempic} we derive an equation for smoothly
ramping tempi to converge and diverge as musical events within a
score, and show how this equation can be used as a stochastic process
to compose previously inaccessible sonorities.

\section{\thesis}
\label{sec:hypercompression-intro}
\newthought{We usually think of compression} in terms of
\emph{reduction}: We use data compression to reduce bit-rates and file
sizes and audio compression to reduce dynamic range. Record labels use
of dynamic range compression as a weapon in the \emph{loudness
  war}\sidenote[][-3cm]{``Loudness War'' is the popular name given to
  the trend of increasing perceived loudness in music recordings.
  Beginning in the 1990s, record labels have attempted to make their
  music louder than the competition, at the expense of audio
  fidelity.}\cite{Deruty2014a}, has resulted in some of today's music
recordings utilizing no more dynamic range than a 1909 Edison
cylinder.\cite{Katz2007} A deeper study of dynamic range compression,
however, reveals more subtle and artistic applications beyond that of
reduction. A skilled audio engineer can apply compression to
improve intelligibility, augment articulation, smooth a performance,
shape transients, extract ambience, de-ess vocals, balance multiple
signals, or even add distortion.\cite{Case2007} At its best, the
compressor is a tool for temporal shaping, rather than a tool for
dynamic reduction.

\thesis expands the traditional model of a dynamic range compressor to
include spatial shaping. While unconventional, spatial processing is a
very natural fit for the compression paradigm. Sound is a medium that
exists in time as well as in space.\sidenote{Converting measurement of
  sound from the cycles per second (in the temporal domain) to
  wavelength (in the spatial domain) is a common objective in
  acoustics and audio engineering practices. See \textit{The Sound
    Reinforcement Handbook} by G. Davis for examples.} The mathematics
and implementation of the Hypercompressor are described in detail in
\autoref{ch:hypercompressor}.

\section{Audio Engineering}
\label{sec:audio-engineering}

Today, just over 50 years after \textit{Formalized Music} was first
published, some of Xenakis' values are widely accepted, while others
are largely ignored. Computers and digital audio workstations (DAWs)
have become ubiquitous in music composition, production, and
performance. The tools for shaping sounds inside DAWs typically use
mathematical language such as frequency, milliseconds, and decibels,
necessitating an understanding of the mathematical representation of
sound. However, these sounds 

Synthesizers for creating musical sounds use simple waveforms,
and parameters for describing intensity and duration, but 

Musicians are increasingly required to understand
the mathematics of music.

Musical synthesizers build complex 

The distinction between musician and engineer is becoming increasingly
indistinct. 



Curious that the simplest tools for engineering audio, 


only exist in spacesound has a direction; sound exists in space. 


As many
musicians before Xenakis have found,


For a theory that was intended to generalize all
of music theory, it is surprising that Xenakis did not include 



At the beginning of the 20th century, there was a blossoming of
complexity, diversity, and invention in contemporary music
composition. Composers such as Debussy, Strauss, Schoenberg and
Stravinsky introduced and devleoped new ways to use harmony and rhythm
that built on the already transofmative works of Wagner in the late
19th century.



Igor Stravin



elements

pointless 

combine studio traditions with art music?

\newthought{In the 6th century B.C.}, Pythagoras discovered that
dividing a resonating string into simple mathematical ratios produced
harmonious musical intervals, while arbitrary ratios produced
dissonance.  His observation is probably the first of the many
explicit parallels between math and music that have been identified
since his time. Today, we describe musical pitches as integers within
a given tuning system. We describe the tuning system with a
mathematical formula that relates frequency to pitch. Musical time,
rhythm and meter are commonly described numerically. Musical
transposition and inversion both mirror mathematical functions and
borrow their names directly from mathematics.


As computers, amplifiers, and electronics become our primary tools for
creating manipulating, and performing music, mathematics and music
necessarily become more interconnected. Nearly every modern musical
recording, broadcast, and stream is the summation of many digital
recordings that have been individually discretized, sampled,
mathematically encoded, decoded, and digitally processed numerous
times before ever reaching our ears.\cite{Case2007} It is tempting to
describe music today as applied mathematics, but doing so betrays a
fundamental quality of music: Musicality does not correspond to % come
                                % back to this. replace musicality
                                % with some
mathematical elegance or precision. A musician will diverge from a
musical score to accomplish a particular artistic objective. A
vocalist does not abruptly change a pitch, but gently and carefully
lands on a pitch. A jazz musician might intentionally play slightly
behind the beat. A classical performer knows how to hold a fermata
just long enough. These intentional human artifacts are characterized
more by a feeling than by a formula.

% Because of the computer's inability..., new musical genres have emerged.
The computer's inability to understand feeling has led to new genres
of music like EDM\sidenote[][-25mm]{EDM (Electronic Dance Music)
  features formulaic and repetitive grooves locked to a temporal grid
  and often incorporates aggressive use of digital pitch correction,
  further exaggerating a robotic quality.}, Black
MIDI\sidenote[][-3mm]{Black MIDI is a musical genre that uses low
  fidelity audio samplers with a large number of MIDI notes over a
  short time. A single three minute Black MIDI track is likely to have
  over 100,000 MIDI notes. The name refers to the solid black
  appearance of the piano score.}, and Demoscene\sidenote{Demoscene
  music celebrates digital synthesis of compositionally complex
  electronic music and audio visualizations, using low level software
  interfaces and including the design and programming of the music
  synthesizers as part of the composition.}, but these styles of music
feature (rather than fix) the inhuman nature of computers. If we want
to integrate a computer into the performance or production of truly
expressive music, we must capture perceived feelings formulaically
and program the computer to reproduce them. This thesis describes
three different, but related projects that confront this challenge from
contrasting perspectives: \refmod, \polytempic, and \thesis.

%\paragraph{Performance}
\thesis was used in the live performance of \textit{De
  L'Exp\'{e}rience}, a new musical work by composer Tod Machover for
Narrator, Organ, and Electronics. During the premier at the Maison
Symphonique de Montr\'{e}al in Canada, \thesis was used to blend the
electronics with the organ and the acoustic space. 
A detailed description of how \thesis featured in this performance is
also discussed in \autoref{ch:hypercompressor}.


\section{Universality}
\label{sec:universality}
At the MIT Media Lab, we celebrate the study and practice of projects
that exist outside of established academic disciplines. The Media Lab
(and the media) have described this approach as interdisciplinary,
cross-disciplinary, anti-disciplinary, or post-disciplinary; rejecting
the clich\'{e} that academics must narrowly focus their studies
learning \textit{more and more about less and less}, and eventually
knowing \textit{everything about nothing}.  The projects described
here uphold the vision of both Xenakis and the Media Lab. Each chapter
documents the movitaions and implementation of a new tool for
manipulating space and sound. Each project draws from an assortment of
fields including music, mathematics, computer science, acoustics,
audio engineering and mixing, sound reinforcement, multimedia
production, and live performance. 

% How can we describe and document a project with such broad subject
% material? Within a single discipline, there is an accepted hierarchy
% of concepts, and we are expected to develop a \emph{deep}
% understanding that penetrates this hierarchy. We expect students to be
% literate in algebra, geometry and calculus before studying
% physics. When we describe a physics problem, we depend on an
% established collection of language, notation, and theory.

% This example reveals the curious tension between breadth and depth:
% The \textit{depth} of a disciplinary approach provides the language
% and abstraction that enable us to describe content and communicate at
% a high level. Depth is essential for solving non-trivial
% problems. However, solutions to the most complex and interesting
% real-world projects always span multiple disciplines. It appears we
% need breadth \emph{and} depth simultaneously. The impact of Iannis
% Xenakis and the success of the Philips Pavilion illustrate the
% efficacy of this approach. 

% \TODO{this thesis is an experiment in breadth and depth}

% One of the goals of this thesis is to
% bridge disciplines by describing the material in a way that is
% accessible to readers that are not experts in all the fields
% involved.


%%% Local Variables:
%%% mode: latex
%%% TeX-master: "CharlesHolbrow_MAS_Thesis"
%%% End:

\clearpage
\chapter{Background}
\label{ch:background}

\paragraph{Early Spatial Music} Western spatial music emerged during
the renaissance period. The earliest published example of spatial
music was by Adrian Willaert in 1550.\cite{Zvonar1999c} The Basilica
San Marco, in Venice, where Willaert was \textit{maestro di capella}
had an interesting feature: Two separate pipe organs facing each other
across the chapel. Willaert took advantage of the organs by composing
music for separate choirs and instrumental groups adjacent the two
organs. Spatially separate choirs soon became a fashion, and gradually
spread beyond Venice, as more and more spatially separated groups were
incorporated into composition. However, interest in spatial
composition declined toward the end of the Baroque period, and was
largely avoided until the Romantic period. Beriloz' \textit{Requiem}
in 1837, Giuseppe Verdi's \textit{Requiem} in 1874, and Mahler's
\textit{Symphony No. 2} in 1895 all feature spatially separated brass
ensembles.

\paragraph{Tempo Acceleration and Deceleration}
Chapter \ref{ch:polytempic} \marginnote{Do not to confuse polytempic
  music with poly\textit{metric} music. The latter is found in West
  African musical tradition, and appear much earlier in Western music
  than polytempi.}  is concerned with oblique, similar, and contrary
tempo accelerations and decelerations in the context of polytempic
(with two or more simultaneous tempi) music.  The tempo indicators
commonly seen today such as \textit{allegro} and \textit{adagio} emerged
during the 17th century in Italy. While these markings partly express
a mood (\textit{gaily}, and \textit{with leisure} respectively) rather
than a strict tempo, they where much easier to follow than the
proportional system (based on tempic ratios such as 3:2 and 5:4) that
they replaced.\cite{Sachs1953} The Intentional use of gradual tempo
changes likely evolved from the unconscious but musical tempo
fluctuations of a natural human performance. We can see examples of
the purposeful manipulation of tempo in the baroque
period. Monteverdi's \textit{Madrigali guerrieri} from 1638, includes
adjacent pieces: \textit{Non havea Febo ancora}, and \textit{Lamento
  della ninfa}. The score instructs to perform the former piece
\textit{al tempo della mano}, in the tactus of the conducting hand,
and the latter \textit{a tempo del'affetto del animo e non a quello
  della mano}, ``in a tempo [dictated by] emotion, not the hand.''
While Monteverdi's use of controlled tempo was certainly not the
first, we are (in particular) interested in gradual tempo changes in
polytempic compositions, which do not appear in western music until
near the beginning of the 20th century.

%\section{Wagner's Gessamtkunstwerk}

\section{20th Century Modernism}
\label{sec:modernism}
As the romantic period was coming to an end, there was a blossoming of
complexity, diversity, and invention in contemporary music. Performers
developed the virtuosic skills required to play the music, composers
also wrote increasingly difficult scores to challenge the
performers.\cite{grout2006} Works by Italian composer Luciano Berio
illustrate the complexity of contemporary music of the time. Beginning
in 1958, Berio wrote a series of works he called
\textit{Sequenza}. Each was a highly of highly technical composition
written for a virtuosic soloist. Each was for a different instrument
ranging from flute to guitar to accordion.  In Sequenza~IV, for piano,
Berio juxtaposes thirty-second note quintuplets, sextuplets, and
septuplets (each with a different dynamic), over just a few
measures. 

\section{Polytempic Music}
\label{sec:background-polytempi}
During the modernist period, composers sought for new ways to use time
and space as compositional elements. Polytempic music was relatively
unexplored. However, traditional music notation is not well equipped
to handle acceleration with precision. The convention is to annotate
the score notes like \textit{ritardando} (gradually slowing),
\textit{ritueno} (abruptly slowing), and \textit{accelerando}
(gradually accelerating) coupled with traditional Italian tempo
markings like \textit{adagio} (slow stately, at ease) and
\textit{allegro} (fast, quickly, bright) to indicate tempo
changes. Rates can be explicitly specified with an M.M.\sidenote{In a
  musical score, M.M. stands for Maelzel's Metronome is accompanies by
  a number specifying the beats per minute.} marking. While rates cam
be quite specific, it is not realistic to expect a performer to be
able to follow an precise mathematical acceleration. This did not stop
modernist composers from finding creative ways to notate surprisingly
precise polytempic compositions using only the conventional notation:
\begin{enumerate}
\item Groups of tuplets layered against a global tempo, as used by
  Henry Cowell (\textit{Quartet Romantic}, 1915-17), and Brian Fernyhough
  (\textit{Epicycle for Twenty Solo Strings}, 1968).
\item Polymeters are notated against a global tempo, and the value of
  a quarter note is the same in both sections, as in Elliott Carter's \textit{Double
    Concerto for Harpsichord and Piano with Two Chamber Orchestras}, 1961
  and George Crumb's \textit{Black Angels}, 1971.
\item Sections are notated without meter. Notes are positioned
  horizontally on the leger linearly according to their position in
  time. Conlon Nancarrow (\textit{Study No. 8 for Player Piano},
  1962), and Luciano Berio (\textit{Tempi Conceriati}, 1958-59).
\item The orchestra is divided into groups, and groups are given
  musical passages with varying tempi. The conductor cues groups to
  begin. Pierre Boulez, \textit{Rituel: In Memoriam Maderna} (1974).
\item One master conductor, directs the entrances of auxiliary
  conductors, who each have their own tempo, and direct orchestral
  sections. This approach was used by Brant Henry in \textit{Antiphony
    One for Symphony Orchestra Divided into 5 Separated Groups} (1953).
\end{enumerate}

\subsection{Charles Ives and The Unanswered Question}
\label{sec:Charles Ives}
Charles Ives' 1908 composition \textit{The Unanswered Question}
incorporates both spatial and polytempic elements. In this piece, the
string section is positioned away from the stage, while the trumpet
solist and woodwind ensemble are on the stage. A dialogue between the
trumpet, flutes, and strings, is written into the music, with the
trumpet repeatedly posing a melodic question \textit{"The Perennial
  Question of Existence''}. Each question is answered by flute
section. The first response is synchronized with the trumpet part, but
subsequent responses accelerate, and intentionally desynchronize with
the soloist. Ives included a note at the beginning of the score which
describes the behavior of the ``answers'':
\begin{quotation}
This part need not be played in the exact time position indicated. It
is played in somewhat of an impromptu way; if there is no conductor,
one of the flute players may direct their playing.

The flutes will end their part approximately near the position
indicated in the string score; but in any case, "The Last Question"
should not be played by the trumpet until "The Silences" of the
strings in the distance have been heard for a measure or two. The
strings will continue their last chord for two measures or so after
the trumpet stops. If the strings shall have reached their last chord
before the trumpet plays "The Last Question", they will hold it
through and continue after, as suggested above.

"The Answers" may be played somewhat sooner after each "Question" than
indicated in the score, but "The Question" should be played no sooner
for that reason.
\end{quotation}
Ives gave the performers license over the temporal alignment, but he
made it clear the parts should not be played together. Following Ives,
other modernist composers also sought new ways to manipulate tempo and
meter. 

\subsection{Gruppen}
\label{sec:gruppen}
Another polytempic example is Karlheinz Stockhausen's \textit{Gruppen}
for three orchestras (1955-57). Parallel tempi that come in and out of
syncronicity is always a challenge with polytempic music, and
Stockhausen found and effective solution. He developed a system of
discrete tempo changes that approximated the logarithmic steps of a
musical scale. Each of the three orchestras was to have it's own
conductor. The conductor would listen for a cue carefully written in
to one of the other sections. That cue would signal to the conductor
to begin beating a silent measure at the new tempo and prepare the new
orchestra to begin playing.
% His invention of the ``tempo scale'' loosely bases they parallel
% tempi in \textit{Gruppen} on the $\sqrt[12]{2}$ ratio of adjacent
% notes in equal temperament.


% Brant, Henry 1978. Antiphonal Responses for 3 Bassoons, Orchestra, 8
% Widely Separated Instruments and Piano Obbligato. New York: Carl
% Fischer.  The eight widely separated instruments are: piccolo,
% clarinet, English horn, trumpet, trombone, {timpani, xylophone or
%   chimes}, {glockenspiel or vibraphone} and harp. Throughout various
% sections, these instruments play repeated phrases at a tempo that is
% different from that of other instruments. The conductor cues their
% entry after which each continues to play without direction,
% uncoordinated with other music. For example, at mark 4 the conductor
% cues the entry of the harp, timpani and vibraphone in relation to the
% bassoons. Here the bassoons have a tempo of MM half note = 56. The
% three entering instruments (whose entry is staggered) have a tempo
% marking of "Slowly, in irregular, free rhythm". At another point, the
% five wind and brass solo instruments are cued by the conductor, one at
% a time, in relation to the bassoons. Here the bassoons play at MM half
% note = 76 and the five entering instruments have a tempo marking of
% "Moderately fast" or "Fast 16ths".

\subsection{Conlon Nancarrow}
\label{sec:conlon-nancarrow}
Conlon Nancarrow is best known for his incredibly complex player piano
scores, and is recognized as one of the first composers to realize the
potential of technology to perform music beyond human capacity. His
later composition for the player piano did incorporate polytempic
accelerations.\cite{Rao2005} Interestingly, he said in an 1977
interview that he was originally interested in electronic music, but
the player piano gave him more temporal control.\cite{Amirkhanian1977}

% Willian Duckworth
% The Time Curve Preludes
% https://en.wikipedia.org/wiki/William_Duckworth_(composer)

\subsection{New Polytempi}
\label{sec:new-polytempi}
The many different approaches to polytempi in modernist music, all
have one thing in common: They all wrestle with syncronicity. Human
performers, are not naturally equipped to play simultaneous tempi, and
composers must find workarounds that make polytempic performance
accessible.

The examples described here exist in one or more of the following
categories:
\begin{enumerate}
\item The music may suggest, multiple tempi, but the beginning and end
  of the individual voices' measures align with each other.
\item The tempo changes are discrete, happening at either measure
  or beat divisions.
\item The tempo changes are somewhat flexible, and  the exact number of
  beats that elapse during a transition varies from one performance to
  another.
\item The tempo acceleration are linear, and align only at simple
  mathematical relationships.
\end{enumerate}
It is non-trivial to rigorously define parallel tempo curves that
accelerate and decelerate continuously relative to each other, and
come into syncronicity at strict predetermined musical points for all
voices. In \autoref{ch:polytempic}, we discuss how existing electronic
and acoustic music approaches this challenge, and derive a
mathematical solution that unlocks a previously inaccessible genre of
polytempic music.

\section{Spatial Developments}
\label{sec:spatial-developments}
\TODO{Berio, Boulez (repons), Stockhausen (Gesang der Yunglinge), etc}


\section{Studio Music}
\label{sec:studio-music}
While 20th century modernist music developed in complexity and
virtuosity, popular music underwent an equally transformative
evolution, developing a different kind of complexity. \TODO{Describe
  Studio Engineering}

\section{Iannis Xenakis}
\label{sec:iannis-xenakis}
These projects build on the work and ideas of Iannis Xenakis, a 20th
century composer, architect, and engineer. He studied music and
engineering at the Polytechnic Institute in Athens, Greece. By 1948,
Xenakis had graduated from the university and moved to France where he
began working for the French architect, Le Corbusier. The job put his
engineering skills to use, but Xenakis also wanted to continue
studying and writing music. While searching for a music mentor, he
approached Oliver Messiaen, and asked for advice on whether he should
study harmony or counterpoint. Messiaen was a prolific French composer
known for rhythmic complexity. He was also regarded as a fantastic
music teacher, and his students included Stockhausen, Boulez.
Messiaen later described his conversation with Xenakis:
\begin{quotation}``I think one should study harmony and
  counterpoint. But this was a man so much out of the ordinary that I
  said: No, you are almost 30, you have the good fortune of being
  Greek, of being an architect and having studied special
  mathematics. Take advantage of these things. Do them in your
  music.''\cite{Service2013}
\end{quotation}
In essence, Messiaen was rejecting Xenakis as a student, but we can
see how Xenakis ultimately drew from his disparate skills in his
compositions. The score for his 1945 composition \textit{Metastasis}
(figure~\ref{fig:metastasis}) resembles an architectural blueprint as
much as it does a musical score.

\begin{figure*}[h]
  \includegraphics[width=\linewidth]{XenakisMetastasis.jpg}
  \caption{Excerpt from Iannis Xenakis' composition,
    \textit{Metastasis} (1954), measures 309-314. This score in this
    image was then transcribed to sheet music for the orchestral
    performance.}
  \label{fig:metastasis}
\end{figure*}

\subsection{The Philips Pavilion}
\label{sec:philips-pavilion-1}
\begin{figure}[h]
  \includegraphics[width=\linewidth]{PhilipsPavilion-TechnicalReview-00.pdf}
  \caption{The Philips Pavilion at the 1958 Brussels World Fair as
    shown in Volume 20 of the \textit{Philips Technical Review}, 1959.}
  \label{fig:philips-pavilion-photo}
\end{figure}
In 1956, Le Corbusier was approached by Louis Kalff (Artistic Director
for the Philips corporation) and asked to build a pavilion for the
1958 World's Fair in Brussels. The pavilion was to showcase the sound
and lighting potential of Philips' technologies. Le Corbusier
immediately accepted, saying:
\begin{quotation}
  ``I will not make a pavilion for you but an Electronic Poem and a
  vessel containing the poem; light, color image, rhythm and sound
  joined together in an organic synthesis.''\cite{Lopez2011} 
\end{quotation}
The final product lived up to Le Corbusier's initial description. It
included:\cite{Lombardo2009}
\begin{enumerate}
\item A concrete pavilion, designed by architect and composer Iannis
  Xenakis
\item \textit{Interlude Sonoire} (later renamed \textit{Concret PH}), a
  tape music composition by Iannis Xenakis, approximately 2 minutes
  long, played between performances, while one audience left the
  pavilion and the next audience arrived
\item \textit{Po\`{e}me \'{E}lectronique}, a three channel, 8 minute
  tape music composition by composer Edgard Var\`{e}se
\item A system for spatialized audio across more than 350 loudspeakers
  distributed throughout the pavilion
\item An assortment of colored lighting effects, designed by Le Corbusier in
  collaboration with Philips' art director, Louis Kalff
\item Video consisting mostly of black and white still images,
  projected on two walls inside the pavilion
\item A system for synchronizing playback of audio and video,
  with light effects and audio spatialization throughout the
  experience
\end{enumerate} 

\paragraph{Role of Iannis Xenakis} During the initial design stage, Le
Corbusier decided that the shape of the pavilion should resemble a
stomach, with the audience entering through one entrance and exiting
out another. He completed initial sketches of the pavilion layout and
then delegated the remainder of the design to
Xenakis.\cite{Clarke2012}

The architectural evolution of the pavilion from Le Corbusier's early
designs (figure~\ref{fig:le-corbusier-sketch}) to Xenakis' iterations
(figure~\ref{fig:xenakis-draw}), illustrates the profound impact that
Xenakis had on the project. An article in the \textit{Philips
  Technical Review}\cite{philips1958} gives a wonderfully detailed
account of Xenakis' process in restructuring the design:\sidenote{\TODO{Clean this section.}}
\begin{enumerate}
\item Xenakis was aware that parallel walls and concave spherical
  walls would both negatively impact audio perceptibility due to repeated
  or localized acoustic reflections.
\item To accommodate musical purpose of the space he decided to
  explore surfaces with varying curvature...
\item 
  \begin{marginfigure}
    \includegraphics{hyperbolic-paraboloid}
    \caption{A ruled surface. For a surface to be considered ``ruled''
      every point on the surface must be on a straight line, and that
      line must lie on the surface. In Xenakis' time, ruled surfaces
      were useful in architecture, because they simplified the
      construction of curved surfaces by using straight beams.}
    \label{fig:ruled-surface}
  \end{marginfigure}...leading him to consider ruled surfaces such as
  the conoid and hyperbolic paraboloid. 
\end{enumerate}
Through this process, we see Xenakis utilizing the skills that he
learned at the Polytechnic Institute and continued to develop while
working with Le Corbusier. He also understood the mathematical
formation of the ruled surfaces that make up the structure. These
surfaces even look familiar to the Metastasis score
(figure~\ref{fig:metastasis}). In his 1963 book, \textit{Formalized
  Music}, Xenakis explicitly states that the Philips Pavilion was
inspired by his work on \textit{Metastasis}.

\begin{figure*}[]
  \includegraphics[width=\linewidth]{LeCorbusierDraw.pdf}
  \caption{Le Corbusier's design sketches for the Philips Pavilion,
    September \textendash{} October, 1956 (\textcircled{c} 2012
    Artists Rights Society, New York/ADAGP, Paris/FLC)}
  \label{fig:le-corbusier-sketch}
\end{figure*}

\begin{figure*}[h]
  % XenakisSketch.pdf or PhilipsDrawings.jpg
  \includegraphics[]{PhilipsDrawings.jpg}
  \caption{Xenakis' early drawings of the Philips Pavilion as
    documented in volume 20 of the \textit{Philips Technical Review}.}
  \label{fig:xenakis-draw}
\end{figure*}

\section{Architecture and Music in Space and Time}
\label{sec:introduction-conclusion}

In \textit{Formalized Music}\cite{xenakis1992formalized}, Xenakis
describes how developments in music theory mimic equivalent
developments in philosophy, mathematics, and the sciences. Plato, for
example, believed that all events transpire as determined by cause and
effect. While Plato and Aristotle both described causality in their
writing, it was not until the 17th century that controlled experiments
and mathematics corroborated the theory.\sidenote{In 1687, Isaac
  Newton published \textit{Philosophi\ae{} Naturalis Principia
    Mathematica} (\textit{Mathematical Principles of Natural
    Philosophy}), in which he compiled the 3 laws of motion that set
  the foundation for the study of \emph{classical mechanics}.}
Similarly, music theory has historically employed causal rules to
describe counterpoint, tonality, and harmonic movement.\sidenote{\TODO{Add example}}

Causality was largely used to describe physical phenomena until the
19th century when statistical theories in physics began to include
probabilistic notions.\sidenote{The Maxwell-Boltzmann distribution,
  which was first derived by James Clerk Maxwell in 1860, describes
  the probability distribution for the speed of a particle within an
  idealized gas. For more see
  \url{http://plato.stanford.edu/entries/statphys-statmech/}} Xenakis
noticed that more contemporary fields like \emph{probability theory}
generalize and expand on the antecedent theories of causality. Xenakis
thought that music composition should naturally follow the progression
that physics did, with music theory generalizing and expanding on
causal rules that had existed previously. Indeed, starting in the late
19th century and early 20th century, composers like Strauss and
Debussy began to bend the existing rules of music theory, composing
music that branched away from the causal and tonal theories of the
time. With the rise of serialism\sidenote{Serialism is a technique for
  musical composition in which instances of musical elements (such as
  pitch, dynamics, or rhythm), are given numerical values. Sequences
  built from the values are ordered, repeated and manipulated
  throughout the composition.}  and indeterminate music\sidenote{In
  music, indeterminacy refers to the use of chance (such as rolling
  dice or flipping coins) as part of the compositional process.},
composers such as Stockhausen, Boulez, John Cage, Aaron Copland, and
B\'{e}la Bart\'{o}k began to use probability and chance in
composition, the same way that physicists were using probability to
describe the material world. 

To Xenakis' mind, serial music was no less causal than the music it
intended to supersede. He described serial music as embodying
``virtually absolute determinism.''\cite{xenakis1992formalized}
Xenakis saw music theory as a sub-set of mathematics and algebra:
While musicians have a different vocabulary, they also use
mathematical principles to describe and compose music. Because Xenakis
understood mathematics as well as music, he was able to identify how
even in serialism and indeterminate music, composers were only
utilizing a small subset of algebraic theory. In his own music,
Xenakis wanted to generalize and expand the causal framework that
musicians and theorists had been using to compose and understand
music, paralleling similar developments in physics and
mathematics. As a reference to \emph{chance}, or \emph{stochos},
Xenakis coined the term \emph{stochastic music} to describe his
development.\sidenote{\TODO{Clarify}}

Xenakis' book, \textit{Formalized Music} gives a verbose explanation
of stochastic music. Some authors have interpreted his description
more explicitly. In \textit{Audible Design}, Trevor Wishart describes
the stochastic process used to compose stochastic music as:
\begin{quotation}
  ``A process in which the probabilities of proceeding from one state,
  or set of states, to another, is defined. The temporal evolution of
  the process is therefore governed by a kind of weighted randomness,
  which can be chosen to give anything from an entirely determined
  outcome, to an entirely unpredictable one.''\cite{Wishart1994}
\end{quotation}
% It could be that the lack of a single clear definition by Xenakis is
% the reason that few composers today identify their work as stochastic
% music.

\paragraph{Xenakis' Reflection} In the Spring of 1976, while defending
his doctoral thesis at the University of Paris, Xenakis emphasized the
relevance of seemingly unrelated disciplines to the creative process. A
translation of his defense includes this statement:
\begin{quotation}
  ``The artist-conceptor will have to be knowledgeable and inventive
  in such varied domains as mathematics, logic, physics, chemistry,
  biology, genetics, paleontology (for the evolution of forms), the
  human sciences, and history; in short, a sort of
  \emph{universality}, but one based upon, guided by and oriented
  toward forms and architectures.''\cite{russolo1986art}
\end{quotation}
From Xenakis' drawings we can deduce that he used the same tools,
skills, and philosophy to imagine and conceive both music and
architecture. His approach elevated both forms and blurred the distinction
between the two. Perhaps if we had kept using pen and paper to design
buildings and write music, the reality today would be closer to the
ideal that he imagined. 

As the ideas that inspired Xenakis and other progressive 20th century
composers were taking root in contemporary music, the culture of
artistic form and composition was already beginning the transition
into the digital domain. There is no reason why digital tools cannot
favor stochastic processes to linearity; there is no reason why
digital tools cannot treat music and architecture as equals. However,
even today, software for composing music still favors static pitches
to glissandi; software for architectural design still favors corners
to curves. Most importantly, the software skills that we use to design
and manipulate space, and the skills that we use to compose music,
mutually exclude each other.

This is where the projects described here make a contribution.  By
drawing from music, mathematics, computer science, acoustics, audio
engineering and mixing, sound reinforcement, multimedia production,
and live performance, we can create tools that allow us to
indiscriminately compose with space and sound.

%%% Local Variables:
%%% mode: latex
%%% TeX-master: "CharlesHolbrow_MAS_Thesis"
%%% End:

\chapter{Spatial Domain: \refmod}
\label{ch:ref-mod}

It was Xenakis' goal for the curved surfaces of the Philips Pavilion
to reduce the sonic contribution of sound reflections as much as
possible.\cite{philips1958} He knew that reflections and the resulting
comb filtering could impair intelligibility and localization of music
and sounds.  The pavilion was to have hundreds of loudspeakers, and an
elaborate custom sequencer electronically selected which speakers
Edgard Var\`{e}se's music played from. However, large concave surfaces
have a of focussing effect on acoustic
reflections,\cite{Vercammen2008} which can result in severe filtering
and phase cancellations. There is evidence that carefully constructed
curved surfaces such as stage shells can be effective for performers
of acoustic music.\cite{DAntonio1991} However, in the context of
loudspeaker playback, the advantages of curved surfaces over flat
(non-parallel) surfaces are ambiguous at best.\cite{Cox2006} When we
hear an acoustic sound reflection off a concave surface, the sound can
arrive at our ears in two possible states:
\begin{enumerate}
\item The path of the sound from the source to the reflecting surface
  to our ears is equidistant for each point on the reflecting
  surface. Ignoring any direct sound, the reflection arrives in-phase,
  and the surface acts as acoustic amplifier of the reflection.
\item The path of the sound from the source to the reflecting surface
  to our ears is slightly different for each point on the surface. All
  the reflections arrive out of phase with each other.
\end{enumerate}
Parabolic microphones\sidenote[][-32mm]{Parabolic microphones use a
  specially designed parabolic reflector that focuses sound arriving
  from one direction on the microphone capsule. Handheld models are
  commonly used in birdsong recording, and on the sidelines of
  football games. They typically have a diameter of two feet or less,
  and can capture sound up to 500 feet away. Due to the size of the
  reflector, commercial parabolic microphones cannot capture sounds
  below approximately 1kHz.}\cite{Davis1989}, and acoustic whispering
chambers\sidenote{A room built such that the walls are angled to
  direct sound from one corner to another. If two people stand in the
  correct spaces in a whispering chamber, they can clearly hear each
  other whispering even though they may be at opposite ends of the
  room.} fall into the first category. Both use surfaces that are
angled to reflect all sound emanating from one direction to converge
at a certain point. If a concave surface reflecting surfaces is not
carefully designed to focus sounds \textit{in phase} it is much more
likely that the sounds will arrive out of phase. The curves of the
Philips Pavilion probably created an \emph{unusual} acoustic space,
rather than a good space for critical listening. However, this may
have been advantageous to the project: Part of the spectacle was
seeing, hearing, and experiencing something completely unprecedented
and unlike anything else.

\section{Composing with Space}
\label{sec:composing-with-space}
If Xenakis had been able to model the reflections and compose them
directly into the piece, what would the tools be like? How can we make
the difference between in phase reflections and out of phase
reflections intuitive?  It we had tools available to compose with
acoustic reflections, how could we use controlled filtering
creatively, and what kind of music, and architectural spaces would we
make? The Xenakis inspired \refmod is tool for experimenting with
architectural acoustic lenses. In several ways, it is an abstraction
from real-world acoustics
\begin{enumerate}
\item It is simplified to two dimensions.
\item It is frequency independent. Real sufaces reflect only
  wavelengths much smaller than the size of the
  reflector.\cite{Zhixin2005}
\end{enumerate}


\begin{figure*}[]
  \includegraphics[width=\linewidth]{refmod/refmod.png}
  \caption[Tempo Transition]{\refmod user interface.}
  \label{fig:basic-tempo-change}
\end{figure*}


%%% Local Variables:
%%% mode: latex
%%% TeX-master: "CharlesHolbrow_MAS_Thesis"
%%% End:


\chapter{\polytempic}
\label{ch:polytempic}

Eliot Carter: Polymetric Modulation. 
Steve Reich: Piano Phase
Paper: realtime representation of 
Paper: Stochos: Software for Real-Time Synthesis of Stochastic Music

John Cage: Chance Music
Karlheinz Stockhausen, Pierre Boulez, Luciano Berio: Aleatoric music



%%% Local Variables:
%%% mode: latex
%%% TeX-master: "CharlesHolbrow_MAS_Thesis"
%%% End:


\chapter{The Hypercompressor}
\label{ch:hypercompressor}

This is how it works

%%% Local Variables:
%%% mode: latex
%%% TeX-master: "CharlesHolbrow_MAS_Thesis"
%%% End:

\clearpage
\chapter{\textit{De L'Exp\'{e}rience}}
\label{ch:experience}

\textit{De L'Exp\'{e}rience} is a composition by Tod Machover in eight
sections for narrator, organ, and electronics. The piece was
commissioned by the \textit{Orchestre Symphonique de Montr\`{e}al}
(OSM) and premiered at the \textit{Maison Symphonique de Montr\'{e}al}
on May 16th, 2015. The text for the piece was taken from the writings
of Michel de Montaigne, the 16th century philosopher known for
popularizing the essay form.  Performers included Jean-Willy Kunz,
organist in residence with the OSM, and narrator Gilles Renaud. A
recording of the performance has been made available
online.\sidenote{\url{http://web.media.mit.edu/~holbrow/mas/TodMachover_OfExperience_Premier.wav}}


\subsection{The Organ}
\label{sec:organ}
The live performance of \textit{De L'Exp\'{e}rience} presented a
unique challenge that fits well with the themes in this thesis. The
acoustic pipe organ can project sound into space unlike any array of
loudspeakers. This is especially true for an instrument as large and
magnificent as the Pierre B\'{e}ique Organ in the OSM concert hall,
which has 6489 pipes and extends to approximately 10 meters above the
stage. Our objective is to blend the sound of the organ with the sound
of electronics.
% The design of the organ is a collaboration between
% Diamond Schmitt Architects and Quebec-based organ manufacturer
% Casavant.

\section{Electronics}
\label{sec:electronics}
The electronics in the piece included a mix of synthesizers,
pre-recorded acoustic cello, and other processed material from
acoustic and electronic sources, all composed by Tod Machover. Prior
to the performance, these sounds were mixed ambisonically:
\begin{enumerate}
\item The cello was placed in front, occupying approximately the front
  hemisphere of our surround sound image.
\item The left and right channel of the electronic swells were panned
  to the left and right hemispheres. However, by default they were
  collapsed to omnidirectional mono (the sound comes from all
  directions, but has no stereo image). The gain of this synth was
  mapped to directionality, so when the synth grows louder, the left
  and right hemisphere become distinct from each other, creating
  an illusion of the sound is growing larger.
\item Additional sound sources are positioned in space, such that each
  has as wide an image as possible, but overlaps with others as little
  as possible.
\end{enumerate}
The overarching goal of this approach was to create a diverse but
interesting spatial arrangement, while keeping sounds mostly panned in
the same spot: movement comes from the warping of the surround image
by the Hypercompressor.

\subsection{Sound Reinforcement}
\label{sec:sound-reinforcement}
Loudspeakers were positioned throughout OSM concert the hall. A number
of factors went into the arrangement: audience coverage, surround
coverage, rigging availability, and setup convenience. All speakers
used were by Meyer Sound.\sidenote{\url{http://www.meyersound.com/}} A
single CQ-2 was positioned just behind and above the narrator to help
localize the image of his voice.  JM-1P speakers on stage left and
stage right were also used for the voice of the narrator, and
incorporated into the ambisonic playback system. Ten pairs of UPJ-1Ps
were placed in the hall, filling in the sides and rear for ambisonic
playback, two at the back of the hall, mirroring the CQ-2s on stage,
four on each of the first and third balconies.  The hall features
variable acoustics, and curtains can be drawn into the hall to
increase acoustic absorption and decrease reverb time. These were
partially engaged, striking a balance: The reduced reverb time
improved the clarity of amplified voice, while only marginally
impacting the beautiful acoustic decay of the organ in the hall. The
show was mixed by Ben Bloomberg. Ambisonic playback and multitrack
recording of the performance was made possible with the help and
expertise of Fabrice Boissin and Julien Boissinot and the Centre for
Interdisciplinary Research in Music Media and Technology (CIRMMT) at
McGill University.

\paragraph{A note on composition, performance, and engineering}
No amount of engineering can compensate for poor composition,
orchestration, or performance. A skilled engineer with the right tools
only can only mitigate shortcomings in a performance. Good engineering
starts and ends with good composition, arrangement and performance. I
have been quite fortunate that all the musicians involved with
\textit{De L'Exp\'{e}rience} at every stage are of the highest
caliber.

\section{Live Hypercompression Technique}
\label{sec:live-hyperc-techn}
During the performance, the encoded ambisonic electronic textures were
patched into the main input of the Hypercompressor, before being
decoded in realtime using the \textit{Rapture3D Advanced} ambisonic
decoder by Blue Ripple
Sound.\sidenote{\url{http://www.blueripplesound.com/products/rapture-3d-advanced}}
Four microphones captured the sound of the organ: two inside and two
hanging in front. The placement of the mics was intended to capture as
much of the sound of the organ as possible, and as little of the sound
of the amplified electronics in the hall. These four microphone
signals where encoded to ambisonics in realtime, and the resulting
ambisonic feed was patched into the side-chain input of the
Hypercompressor. In this configuration, the organ drives the
spatialization of the electronic sounds. By ambisonically panning the
organ microphones, we can control how our electronics are
spatialized. After some experimentation we discovered the best way to
apply the Hypercompressor in the context of \textit{De
  L'Exp\'{e}rience}. When the organ was played softly, the sound of
the electronics filled the performance hall from all directions. As
the organ played louder, the electronic textures dynamically warped
toward the organ in the front of the concert hall. The spatial and
timbral movement of the electronics together with the magnificent (but
stable) sound of the the organ created a unique blend that would be
inaccessible with acoustic or electronic sounds in isolation.

\begin{figure*}[]
  \includegraphics[width=\linewidth]{DressRehersal.jpg}
  \caption{The Pierre B\'{e}ique Organ in the OSM concert hall during
    a rehearsal on May 15th, 2015. Approximately 97\% of the organs'
    6489 pipes are out of sight behind the woodwork. Photo credit: Ben
    Bloomberg}
  \label{fig:le-corbusier-sketch}
\end{figure*}


%%% Local Variables:
%%% mode: latex
%%% TeX-master: "CharlesHolbrow_MAS_Thesis"
%%% End:

\clearpage
\chapter{Discussion and Analysis}
\label{ch:analysis}
In the previous chapters we explored three new tools for creating and
processing music, including their motivations and implementations.
\polytempic in \autoref{ch:polytempic} proposed a mathematical
approach for composing previously inaccessible polytempic music. The
\refmod in \autoref{ch:ref-mod} introduced an interface for quickly
sketching abstract architectural and musical ideas.  Chapters
\ref{ch:hypercompressor} and \ref{ch:experience} described the
motivations for an implementation of a new technique for moving music in
space and time. Each of these projects builds on Iannis Xenakis'
theory of \textit{stochastic music} and incorporates elements from
other disciplines, including mathematics, computer science, acoustics,
audio engineering and mixing, sound reinforcement, multimedia
production, and live performance.

This final chapter discusses how each project succeeded, how each
project failed, and how future iterations can benefit from lessons
learned during the development process.

\section{Evaluation Criteria}
\label{sec:eval-criteria}
To evaluate a project of any kind, it is helpful to begin with a
purpose, and then determine the granularity and scope of the
evaluation.\cite{Saltzer2009} We might evaluate a music recording for
audio fidelity, for musical proficiency of the artist, for emotional
impact or resonance, for narrative, for technological inovation, for
creative vision, or for political and historical insight. Similarly,
we can evaluate the suitability of an analog to digital converter
(ADC) for a given purpose. If our purpose is music recording, we might
prefer different qualities than if our purpose is electrical
engineering. A recording engineer might prefer that the device impart a
favorable sound, while an acoustician may prefer that the device be as
neutral as possible.

In the evaluation of a music recording, and the evaluation of an ADC,
we concern ourselves with only the highest level interface: When
evaluating a music recording, we listen to the sound of the
recording, but we do not evaluate the performance of the ADC used to
make the recording. Evaluation is simplified when we consider fewer
levels of abstraction.

Stochastic music theory is a \textit{vertical integration} of
mathematics, the physics of sound, psychoacoustics, and music. The
theory of stochastic music begins with the lowest level components of
sound and ends with a creative musical product. What is a reasonable
perspective from which to evaluate stochastic music? From the
perspective of listening or performing the music? From the
perspective of a historian, evaluating the environment that led to the
composition or studying the impact on music afterwards?  Should we
try to make sense of the entire technology stack, or try to evaluate
every layer of abstraction individually? Somehow, between the
low-level elements of sound and a musical composition or performance,
we transition from what is numerically quantifiable to what we can
only attempt to describe.

In my evaluation, I focus on two qualities. First, I study how each
project achieved its original objectives and how it fell short. Second,
I consider how each project can influence or inspire future
iterations. I avoid comparative analysis or evaluation based on any
kind of rubric. Instead, I evaluate the results of the project
according to its own motivations and historical precedents.

% Does the unit have an appropriate number and type of audio outputs
% suit our needs (such as TRS, XLR, and USB)? Does it perform
% reliably? Does it have a neutral or favorable impact on the sound?
% The results of our evaluation will depend on our purpose. If the
% purpose of the device is for playback of audio in a live performance
% context, the best choice will be different than if we want to use
% the unit for mixing in a recording studio.

\section{\polytempic}
This chapter presents a very pure and elegant solution to a very
complex problem. But is it important? Is it a significant improvement
on the existing techniques presented in section
\ref{sec:background-polytempi}? If a performer cannot play precise
tempo curves anyway, what is this actually for?

Western polytempic music as defined in \autoref{ch:polytempic} has
existed for only slightly over one century, and there is certainly
room for new explorations. The oldest example of Western polytempic
music is by Charles Ives in his 1906 piece, \textit{Central Park in
  the Dark}.\cite{Greschak2003} In the piece, the string section
represents nighttime darkness, while the rest of the orchestra
interprets the sounds of Central Park at night. Beginning at measure
64, Ives leaves a note in the score, describing how the orchestra
accelerates, while the string section continues at a constant tempo:
\begin{quotation}
  From measure 64 on, until the rest of the orchestra has played
  measure 118, the relation of the string orchestra's measures to
  those of the other instruments need not and cannot be written down
  exactly, as the gradual accelerando of all but the strings cannot be
  played in precisely the same tempi each time.
\end{quotation}
Ives acknowledges that there is no existing notation to exactly
describe the effect that he wants, and that musicians are not capable
of playing the transition in a precise way. In this example, it is not
important that the simultaneous tempi have a precise rhythmic
relationship. Ives' use of parallel tempi is a graceful one. He
achieves a particular effect without requiring the musicians to do
something as difficult as accelerate and decelerate relative to each
other, and then resynchronize at certain points.

All polytempic compositions must grapple with the issue of
synchronicity, and many demand more precision than \textit{Central
  Park in the Dark}. Stockhausen's \textit{Gruppen} uses polytempi
very aggressively, going to great lengths to ensure that the three
orchestras rhythmically synchronize and desynchronize in just the
right way. If Stockhausen had been able to control the synchronicity
of the tempo precisely, it seems likely that he would have wanted to
try it.

Some music (and perhaps stochastic music in particular) may be more
interesting or influential from a theoretical perspective, than for
the music itself in isolation. It could be that the possibilities
unlocked through the equations derived in \autoref{ch:polytempic} are
not different enough from the approximations used by Nancarrow and
Cage or that it is unrealistic to direct performers to play them
accurately enough to perceive the difference.

However it is surprising that current digital tools for composition do
not let us even \emph{try} \polytempic. If we want to hear what tempo
transitions like the ones describe here sound like using digital
technology, there is no software that lets us do so, and we are still
forced to approximate. Audio programming languages like Max and
SuperCollider let us code formulaic tempi into our compositions, but
equations like the ones derived here are still required. I could not
find any technique that lets us create swarms of tempo
accelerations that fit the constraints described in
\autoref{ch:polytempic}, or any musical example that proposed to have
found another solution.

For some cases approximation is perfectly acceptable. If a musician
is incapable of playing the part, we are also likely incapable of
hearing the subtleties that distinguish an approximation from a
perfect performance. However, if we want large collections of
simultaneous polytempi, like the ones shown in
figures~\ref{fig:polytempic-transition}
and~\ref{fig:polytempic-transition-3}, the approximations possible
with transcriptions, or the approximations of an unassisted human
performers, are not precise enough.

\paragraph{Next Steps} Bryn Bliska's composition (linked in
section~\ref{sec:composition}) is a good starting point for future
explorations, but it was composed with an earlier version of the
polytempic equation that did not allow for large swarms of tempi. In
its current state, the polytempic work described in this thesis is
just a beginning. We have not yet tried to compose a piece that
fully incorporates \polytempic. 

Equations alone do not make a musical instrument, and composition is
difficult without a musical interface. There are a few modern
examples of polytempic projects (see \autoref{ch:polytempic}), but I
could not find any examples of interfaces for composing with
coordinated mass of tempi. The most exciting direction for
this project is the creation of new musical interfaces for composing
and manipulating stochastic tempo swarms. 


\section{\refmod}
This project provides a single abstract interface that approaches
composition of space (architecture) and the composition of music at
the same time. The forms that it makes are familiar from the ruled
surfaces seen in Xenakis' compositions and early sketches of the
Philips Pavilion. From a musical perspective, we can think of the x
and y axes representing time and pitch. From an architectural
perspective the canvas might represent the floor plan of spaces we are
designing.

While it is interesting to switch our perspective between the two
modes, there is not a clear connection from one to the other. A
carefully designed surface or reflection in one mode would be quite
arbitrary in the other mode. The reason that the interface is capable
of working in both modes is because it is so abstract that it does
not commit to one or the other. This is not a complete failing: The
tool was really designed to be a brainstorming aid at the very
beginning of the design process. It can be much simpler and quicker to
use than proper architectural software as a means of creating
abstract shapes, similar to sketching on paper, before turning to
specialized software for more detailed design.

\paragraph{Curves, Constraints, and Simplicity} Despite the
limitations of this project, the parts that worked well do form a
strong base for future iterations. There is something simple and
\textit{fun} about the user interface. There is only one input action;
dragging a control point. It is immediately clear what each control
point does when it is moved. It is easy to not even notice that there
are five different types of control points and each has slightly
different behavior. It is very intuitive to adjust a reflection
surface such that the red beams \textit{focus} on a certain point, and
then re-adjust a reflection surface so that they diverge
chaotically. There is something fascinating about how the simple
movements intuitively produce coordinated or chaotic stochastic
results.

The red ``sound lines'' have three degrees of freedom: position,
direction, and length. We can point the rays in any direction we like,
but their movement is somewhat constrained.  The projection angle is
locked to 30 degrees and the number of beams is always eight, and
most of the flexibility from the interface comes from the reflective
surfaces.

\paragraph{Stochastic by Default} The \refmod interface makes it easier
to draw a curving reflective surface than a straight one. If you make
a special effort, it is possible to make one of the surfaces straight,
but just like drawing a line on a paper with a pen, curved surfaces
come more naturally. The curves in the \refmod do not come naturally
because they are following an input gesture like most ``drawing''
interfaces, but because of the simple mathematics in the of the Bezier
curves. If we consider the red lines to be notes on a time/pitch axis,
the default interpretation is stochastic glissandi rather than static
pitches. Most musical software assumes static pitches by default and
most architectural software assumes straight lines.

\paragraph{Next Steps}
The obvious next steps for this project involve correcting the
shortcomings described above. It could be made to work in three
dimensions, and model precise propagation of sound rather than a very
simplified abstraction: It could become a proper acoustical
simulator. Another possibility is turning it into a compositional or
performative musical instrument where we can hear the stochastic
glissandi in realtime. These options are not necessarily mutually
exclusive, but as the interface becomes tailored to a more specific
application, our ability to think about the content as abstract
representations also breaks down. The ideal of software that is
equally well-equipped to compose music and to imagine architectural
spaces is probably unrealistic. 

Any visual representation of music is quite abstract, and different
visual representations can encourage us to think about music in new
and unusual ways. For example each red line can be considered pitch,
but it can also be considered its own time axis. By calculating the
red paths, we can creating many time axes that follow similar but
slightly different trajectories. Alternatively, each red line can be thought of as a
time axis for an individual pitch. When the lines collide with a
curved surface after slightly different lengths, it represents an
arpeggiated chord. In contrast, a non-arpeggiated chord is represented
when the red lines all collide with a surface after traveling
identical distances. The abstract nature of this interface leaves room
for our imagination to interpret unexpected new musical
possibilities. 


\section{\thesis}
The design and development of \thesis happened in parallel with
pre-production for \textit{De L'Exp\'{e}rience}, and the
Hypercompressor was, in part, tailored to the needs of a somewhat
unique situation. The resulting project leaves significant design
questions surrounding ambisonic dynamic range compression unanswered.
For example: What is the best way to detect and attenuate a region on
our surround sphere that is an unusual or elongated shape?  Should the
compressor attempt to attenuate the narrow region only?  Should we
attenuate the center of the region more than the edge?


When a region of our surround sound image exceeds the
Hypercompressor's threshold, the compressor warps the surround image
in addition to attenuating the region where the threshold overage
occurred. This makes sense for side-chain compression, but is less
applicable to standard compression. We could have chosen only
warping, or only attenuation, each of which represents its own
compromise:
\begin{itemize}
\item We could simply warp all sounds away from a region that exceeds
  the compression threshold without attenuating them at all. However,
  doing so would increase the perceived level of the sound coming from
  the opposite direction. We also run the risk of creating sonic
  ``ping-pong'' of sounds arbitrarily panning. This can sound
  exciting, but quickly becomes a contrivance or gimmick. 
\item If we simply attenuate a region that exceeds the threshold, we
  are not taking advantage of the opportunities provided to us by
  surround sound in the first place. In side-chain mode, we risk
  hiding a compressed sound completely when we could simply warp that
  region of the surround field to a location where it can be heard
  more clearly.
\end{itemize}
The current implementation also does not handle the case when two
separate regions of the surround field both exceed the threshold.

\paragraph{\textit{De L'Exp\'{e}rience}}
The main goal of using the Hypercompressor was to blend the electronic
textures with the sound of the Pierre B\'{e}ique organ in Tod
Machover's composition. The chosen approach was to give the
electronics a sense of motion that the organ (whose sound is
awe-inspiring, but also somewhat static) cannot produce; thus the
electronics can be heard moving \emph{around} the sound of the organ,
rather than being required to compete with the sound of the organ.
The first attempt at this goal, however, did not go as planned.

The electronics were mixed to occupy as much of the surround sound
sphere as possible, filling the entire room with sound.  My original
idea was to spatially separate the organ and electronics by connecting
them to the Hypercompressor in side-chain mode.  When the organ was
playing it would \emph{push} the sound of the electronics to the back
of the room, making it easier to hear both timbres without either
masking the other.  During the \textit{De L'Exp\'{e}rience} rehearsal,
this was the first approach I tried, but the resulting surround
texture had a different problem: The sound of the organ and the sound
of the electronics were \emph{too} separate. They did not blend with
each other in space, but existed as two clearly distinct sources. I
arrived at the solution described in \autoref{ch:hypercompressor} only
after first trying the exact opposite of the final approach. While I
had to revise my strategy during the rehearsal I consider the
Hypercompressor to have aided the blending of the organ and
electronics especially well. It is important to note that the
beautiful blend of sounds that we achieved would have been possible
without many other contributing factors, such as the expert
composition of the electronic textures.


\section{Stochos}
\thesis is a complete realization of a musical idea. Beginning with an
objective and a mathematical foundation, we designed and built a custom
software implementation and applied it in a live performance
context. A study of the process has revealed what is probably the
greatest strength of stochastic music theory: The vertical integration
of the theory of sound and music lets us study music from a privileged
perspective, while the controlled chance built into the system helps
us to uncover possibilites that could not be found with conventional
means. In the case of \thesis, we move sounds in space based on matrix
transforms, that are themselves driven by the controlled chance of a
random performance. The angular position of our sounds are defined by
both explicit mathematical formulas and the unpredictable qualities of
live performance.

From a broad perspective all three projects emerged ``by chance'' in
the same way. Each one is the result of musical exploration in a space
that indiscriminately draws from mathematics, computer science,
acoustics, audio engineering and mixing, sound reinforcement,
multimedia production, and live performance. By treating all these
disciplines as components of music theory, we discover new musical
patterns and possibilities for shaping sound in time and space.

% The positioning sounds in space 
% We have discussed the advantages and disadvantages of warping and
% attenuation as techniques for handling surround sound
% compression. Attenuation is the approach used most mono and stereo
% recordings, while surround sound lets us explore warping as an
% alternative. One obvious next step is to build a version that simple
% lets us toggle (or adjust continuously) between a warping and
% attenuation.

\clearpage
\chapter*{Epilogue}
\label{ch:epilogue}

% The grant was 99510\EUR{}
In 2004, the Culture 2000 Programme, created by the European Union
approved a grant to an Italian multimedia firm for a project called
Virtual Electronic Poem (VEP).\cite{eu2004} The project proposed the
creation of a virtual reality experience in which users could enter a
simulated version of the famous Phillips Pavilion. While developing
the VEP, the design team went through the archives of Xenakis, Le
Corbusier, and Philips, uncovering every relevant bit of information
in order to make the experience as real as
possible.\cite{Lombardo2009}

Virtual reality technology changed so much between 2004 and 2015 that
reviving the VEP project today would likely involve an additional
multimedia archeology expedition as intensive as the first: It would
probably be easier (and more effective) to start from scratch using
the original documentation. A common problem with multimedia
performances is that technology changes so fase that it quickly
becomes very difficult to restore even moderately recent
projects.\cite{Lombardo2006} In contrast, the mathematical language
that Xenakis used to describe his work is as well-established as the
language of Western music notation, and for this reason we have a
surprisingly thorough understanding of his music today. It is my hope
that the documentation in this thesis will provide an equally
dependable and enduring a description of the process of modern musical
composition.



% Having mixed a bunch of stuff, for different contexts, the different
% panning strategies in discrete channel, ambisonics, don't are not
% better or worse. Stockhausen didn't really didn't need anything other than
% discrete channel panning for Gesang. In fact the piece holds up very
% well in stereo. I do think that there is a difference between actual
% surround and 5.1, the latter resembling dual stereo more than actual
% surround. 

% welp, not sure how well it actually
% worked. And I don't expect electronic instrument to ever replace
% acoustic instruments, although they may eventually equal them in
% expressivity. Initially thought it would work backwards

%%% Local Variables:
%%% mode: latex
%%% TeX-master: "CharlesHolbrow_MAS_Thesis"
%%% End:


\backmatter
\clearpage
\chapter*{Acknowledgements}
\label{ch:acknowledgements}
\begin{fullwidth}
  \noindent Thanks to Tod Machover for welcoming me into the Opera of
  the Future, for unending support, encouragement, and mentorship; for
  sharing your process, and for support integrating \thesis into
  \textit{Of Experience}.

\vspace{5mm}
\noindent Thanks to my readers James Andy Moorer and Joe Paradiso for
kind, and expert input, support, and guidance.

\vspace{5mm}
\noindent Thanks to Professor Alex Case for inspiring my love
for music, audio engineering, and illuminating the magical subtleties
of dynamic range compression, and for (only slightly begrudgingly)
writing me 1000 recommendations letters. No one is better suited to be
President of the Audio Engineering Society.

\vspace{5mm}
\noindent Thanks to Wonshik Choi and Niyom Lue for your infinite
patience, guidance, and for welcoming me to MIT in 2008. Only you
could have taught a music major to enjoy linux, DSP, and spectroscopy.

\vspace{5mm}
\noindent Thanks to Markus Phillips and Shawn Drost for directly and indirectly
giving me confidence as a software developer.

\vspace{5mm}
\noindent Thanks to my UMass Lowell Piano teachers for taking chance
with me, and putting up with me for four years. Anthony Mele,
Elizabeth Skavish, Bonnie Anderson, and Thomas Stumpf - You believed
in me before I did. I'm probably the only student ever who was lucky
enough to study with all four of you.

\vspace{5mm}
\noindent  Thanks to Gene Atwood for being considerate of everyone, and showing
me how important that is\ldots{} And for screaming in to a microphone when
I needed some screams.

\vspace{5mm}
\noindent Thanks to my roommates at The Red House, and Brainerd
Contextual Cooperative for love and support and home. For feeding me
lots of meals, for perspective, and for tolerating my student
lifestyle.

\vspace{5mm}
\noindent Thanks to my team at the MIT Media Lab. Ben Bloomberg for
being my peer and my mentor at the same time. Thanks for inviting me
into the Opera group in 2008 and again in 2014, teaching me a
tremendous amount about audio, and using your live sound prowess to
make \textit{De L'Exp\'{e}rience} sound amazing live. Bryn Bliska, for
conceiving Tempo Toy, which I only slightly stole and turned into
stochastic polytempic modulation, for letting me work out the maths,
and for humility, warmth and musical brilliance. Thanks to Rebecca
Kleinberger for infinite support and insight into everything from math
to hedgehogs to humans. David for peace, intensity, and robots. Akito
for your inspiring attitude, wisdom, projects, and humility. Kelly for
unending support, consideration, and sense of humor. Simone
for taking the ``im'' out of impossible (over and over and over
again).

\vspace{5mm}
\noindent Thanks to Helen Corless for being amazing supportive even
when I am in the absolute pits of grad student existence. Thank you
for always reminding me what music is really about, and for
challenging me like no one else can. Your kindness and wisdom make
the world a better place every day.

\vspace{5mm}
\noindent Thanks to my grandparents for leading by example, and teaching
kindness and dedication, and for endless support in education.

\vspace{5mm}
\noindent Thanks to my wonderful siblings: Hilary, Giles, and Felicity
for always inspiring me with kindness, honesty, and wisdom.

\vspace{5mm}
\noindent And thanks to my parents, Gwen and Mark for forcing me to get an
education before I was wise enough to know I wanted one. Thank you for
all your love and support and everything forever.
\end{fullwidth}

%%% Local Variables:
%%% mode: latex
%%% TeX-master: "CharlesHolbrow_MAS_Thesis"
%%% End:


\bibliography{library}
\bibliographystyle{plainnat}

\end{document}


%%% Local Variables:
%%% mode: latex
%%% TeX-master: t
%%% End:
