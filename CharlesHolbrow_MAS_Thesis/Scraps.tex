\section{Structure}
\label{sec:structure}

n\TODO{I'll fill this out more as I progress, but this is basically the
  chapter structure I'm imagining. Building the software likely won't
  have enough content for a whole chapter, so I may merge it with the
  math chapter}
\begin{enumerate}
\item Introduction and  Historical Context % context primes for motivation
\item Motivation
\item Framework for multimedia documentation Approach/Computer Systems
\item Describe the Mathematics. % Note that Wikipedia states that
                                % Xenakis pioneered the use of
                                % Set-theory in music. this is a nice
                                % tie in to Set Builder
                                % Notation, which when seen is ungoogleable
\item Build the Software
\item Live multimedia performance
\item Concert playback recording. 
\item Conclusion/Educational Implications
\end{enumerate}


\section{Virtual Electronic Poem}
\label{sec:vep}

\newthought{In 2004}, the Culture 2000 Programme created by the 
European union approved a 99510\EUR{} grant to an Italian Multimedia 
firm for a project called Virtual Electronic Poem (VEP)\cite{eu2004}. 
The project proposed to create a virtual reality simulation in which 
users could experience rendered audio and video of the famous
Gesamtkunstwerk created for the Phillips Pavilion at the 1958 World
Expo in Brussels. 

The goal of the project is included in the Culture 2000 Programme
report: The project will reproduce the experience created by the
Philips pavilion at the Brussels Expo Universelle in 1958.

Give Details? Explain all the parts that were involved with the
original pavilion, explain all the resources that were reviewed to
figure out what the thing actually looked like.

This project was so involved, that the principal investigator coined
the term "Archeology of Multimedia" to describe the experience of
recovering\TODO{cite}.

Virtual reality technology has changed enough between 2004 and 2015,
that reviving the VEP project would probably take an additional
multimedia archeology expedition.

The problem of preservation described above informs the structure and
strategy in this thesis. 

Explain the need for audio systems as computer systems? Cite
Granularity. Describe what fields I draw from. Describe how fields can
go deep, but Anti Disciplinary is unbounded. Describe how this could
only happen at the media lab. Detail how I will granular-ize?

\section{Motivation}
\label{sec:motivation}

  - Sal Khan's talk about students who get stuck. 
  - Talk about the expectations for this paper - what should you
    already understand? Basic Audio theory. 
  - specialization makes it hard for different fields to communicate
  with each other and learn from each other.
  - Think about documentation in terms of Systems, granulatiry
  - Noticed how very simple concepts are portrayed 
  - Leave behind bridges between disciplines
  - set builder notation, and un-google-able
  questions. IE. Gesamtkunstwerk is googleable, and does not need a
  citation.
  - Hypertext impacts retention and comprehension

% Hypertext link 1:
% https://scholar.google.com/scholar?hl=en&as_sdt=0%2C22&sciodt=0%2C22&cites=10588406244047104644&scipsc=&as_ylo=2007&as_yhi=2007

% Hypertext link 2: 
% Cognitive load in hypertext reading: A review Diana DeStefano
% http://www.sciencedirect.com/science/article/pii/S0747563205000658

\chapter{Compression Overview}
\label{cha:compression-overview}

\paragraph{Compression and Hypercompression} A traditional
\marginnote{Be aware of the diference between audio data compression
  and audio dynamic range compression. Data compression is the
  practice of reducing the filesize of digitally encoded audio data. A
  dynamic range compressor is one tool used by audio engineers to
  parametrically manipulate the amplitude of an audio signal. In this
  thesis, \textit{compression} refers to \textit{dynamic range
    compression}.}
dynamic range compressor is one of the most powerful and flexible
tools in the audio engineer's toolkit.  A basic compressor can be
thought of as an automatic volume control that simply reduces the
level of an audio signal when the signal exceeds a threshold. 

%%% Local Variables:
%%% mode: latex
%%% TeX-master: "CharlesHolbrow_MAS_Thesis"
%%% End:
