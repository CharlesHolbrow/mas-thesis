\section{killed from RefMod}
Parabolic microphones\sidenote[][-32mm]{Parabolic microphones use a
  specially designed parabolic reflector that focuses sound arriving
  from one direction on the microphone capsule. Handheld models are
  commonly used in birdsong recording, and on the sidelines of
  football games. They typically have a diameter of two feet or less,
  and can capture sound up to 500 feet away. Due to the size of the
  reflector, commercial parabolic microphones cannot capture sounds
  below approximately 1kHz.}\cite{Davis1989}, and acoustic whispering
chambers\sidenote{A whisper chamber, or whisper gallery is a room
  built such that the walls are angled to direct sound from one corner
  to another. If two people stand in the correct spaces in a
  whispering chamber, they can clearly hear each other whispering even
  though they may be at opposite ends of the room.} fall into the
first category. Both use surfaces that are angled to reflect all sound
emanating from one direction to converge at a certain point. If a
concave surface reflecting surfaces is not carefully designed to focus
sounds \textit{in phase} it is much more likely that the sounds will
arrive out of phase. The curves of the Philips Pavilion probably
created an \emph{unusual} acoustic space, rather than a good space for
critical listening. However, this may have been advantageous to the
project: Part of the spectacle was seeing, hearing, and experiencing
something completely unprecedented and unlike anything else.

\section{Yanked from Introduction}
\label{sec:yank-from-intr}

Nearly every modern musical recording, broadcast, and stream is the
summation of many digital recordings that have been individually
discretized, sampled, mathematically encoded, decoded, and digitally
processed numerous times before ever reaching our ears.\cite{Case2007}

It is tempting to describe music today as applied mathematics, but
doing so betrays a fundamental quality of music: Musicality does not
correspond to mathematical elegance or precision. A musician will
diverge from a musical score to accomplish a particular artistic
objective. A vocalist does not abruptly change a pitch, but gently and
carefully lands on a pitch. A jazz musician might intentionally play
slightly behind the beat. A classical performer knows how to hold a
fermata just long enough. These intentional human artifacts are
characterized more by a feeling than by a formula.

% Because of the computer's inability..., new musical genres have emerged.
The computer's inability to understand feeling has led to new genres
of music like EDM\sidenote[][-25mm]{EDM (Electronic Dance Music)
  features formulaic and repetitive grooves locked to a temporal grid
  and often incorporates aggressive use of digital pitch correction,
  further exaggerating a robotic quality.}, Black
MIDI\sidenote[][-3mm]{Black MIDI is a musical genre that uses low
  fidelity audio samplers with a large number of MIDI notes over a
  short time. A single three minute Black MIDI track is likely to have
  over 100,000 MIDI notes. The name refers to the solid black
  appearance of the piano score.}, and Demoscene\sidenote{Demoscene
  music celebrates digital synthesis of compositionally complex
  electronic music and audio visualizations, using low level software
  interfaces and including the design and programming of the music
  synthesizers as part of the composition.}, but these styles of music
feature (rather than fix) the inhuman nature of computers. If we want
to integrate a computer into the performance or production of truly
expressive music, we must capture perceived feelings formulaically
and program the computer to reproduce them. This thesis describes
three different, but related projects that confront this challenge from
contrasting perspectives: \refmod, \polytempic, and \thesis.

\section{Random Ideas}
\label{sec:random-ideas}

Organizing complexity is now more important than ever. Software
development. Music Theory. Accomodating short attention spans.

Combinatory Logic is another notation, along with set builder
notation. See Also:
https://en.wikipedia.org/wiki/Quantifier_(logic)#Mathematics

Real coordinate space  - https://en.wikipedia.org/wiki/Real_coordinate_space


\section{Structure}
\label{sec:structure}

n\TODO{I'll fill this out more as I progress, but this is basically the
  chapter structure I'm imagining. Building the software likely won't
  have enough content for a whole chapter, so I may merge it with the
  math chapter}
\begin{enumerate}
\item Introduction and  Historical Context % context primes for motivation
\item Motivation
\item Framework for multimedia documentation Approach/Computer Systems
\item Describe the Mathematics. % Note that Wikipedia states that
                                % Xenakis pioneered the use of
                                % Set-theory in music. this is a nice
                                % tie in to Set Builder
                                % Notation, which when seen is ungoogleable
\item Build the Software
\item Live multimedia performance
\item Concert playback recording. 
\item Conclusion/Educational Implications
\end{enumerate}


\section{Virtual Electronic Poem}
\label{sec:vep}



The problem of preservation described above informs the structure and
strategy in this thesis. 

Explain the need for audio systems as computer systems? Cite
Granularity. Describe what fields I draw from. Describe how fields can
go deep, but Anti Disciplinary is unbounded. Describe how this could
only happen at the media lab. Detail how I will granular-ize?

\section{Motivation}
\label{sec:motivation}

  - Sal Khan's talk about students who get stuck. 
  - Talk about the expectations for this paper - what should you
    already understand? Basic Audio theory. 
  - specialization makes it hard for different fields to communicate
  with each other and learn from each other.
  - Think about documentation in terms of Systems, granulatiry
  - Noticed how very simple concepts are portrayed 
  - Leave behind bridges between disciplines
  - set builder notation, and un-google-able
  questions. IE. Gesamtkunstwerk is googleable, and does not need a
  citation.
  - Hypertext impacts retention and comprehension

% Hypertext link 1:
% https://scholar.google.com/scholar?hl=en&as_sdt=0%2C22&sciodt=0%2C22&cites=10588406244047104644&scipsc=&as_ylo=2007&as_yhi=2007

% Hypertext link 2: 
% Cognitive load in hypertext reading: A review Diana DeStefano
% http://www.sciencedirect.com/science/article/pii/S0747563205000658

\chapter{Compression Overview}
\label{cha:compression-overview}

\paragraph{Compression and Hypercompression} A traditional
\marginnote{Be aware of the diference between audio data compression
  and audio dynamic range compression. Data compression is the
  practice of reducing the filesize of digitally encoded audio data. A
  dynamic range compressor is one tool used by audio engineers to
  parametrically manipulate the amplitude of an audio signal. In this
  thesis, \textit{compression} refers to \textit{dynamic range
    compression}.}
dynamic range compressor is one of the most powerful and flexible
tools in the audio engineer's toolkit.  A basic compressor can be
thought of as an automatic volume control that simply reduces the
level of an audio signal when the signal exceeds a threshold. 



\urldef{\refvisone}\href{http://web.media.mit.edu/~holbrow/mas/reflections/?q=%7B%22mirrors%22%3A%5B%5B%22Path%22%2C%7B%22applyMatrix%22%3Atrue%2C%22segments%22%3A%5B%5B%5B330%2C60%5D%2C%5B0%2C0%5D%2C%5B-239%2C185%5D%5D%2C%5B319%2C424%5D%5D%2C%22strokeColor%22%3A%5B0%2C0%2C0%5D%2C%22strokeWidth%22%3A2%7D%5D%2C%5B%22Path%22%2C%7B%22applyMatrix%22%3Atrue%2C%22segments%22%3A%5B%5B%5B1035%2C436%5D%2C%5B0%2C0%5D%2C%5B-135%2C113%5D%5D%2C%5B1123%2C534%5D%5D%2C%22strokeColor%22%3A%5B0%2C0%2C0%5D%2C%22strokeWidth%22%3A2%7D%5D%5D%2C%22sounds%22%3A%5B%5B%22Path%22%2C%7B%22applyMatrix%22%3Atrue%2C%22segments%22%3A%5B%5B1047%2C123%5D%2C%5B1048%2C145%5D%5D%2C%22strokeColor%22%3A%5B0.6%2C0.6%2C0.6902%5D%7D%5D%2C%5B%22Path%22%2C%7B%22applyMatrix%22%3Atrue%2C%22segments%22%3A%5B%5B411%2C176%5D%2C%5B480%2C136%5D%5D%2C%22strokeColor%22%3A%5B0.6%2C0.6%2C0.6902%5D%7D%5D%5D%7D}{Link}


%%% Local Variables:
%%% mode: latex
%%% TeX-master: "CharlesHolbrow_MAS_Thesis"
%%% End:
